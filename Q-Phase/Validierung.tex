\documentclass[a4paper,12pt]{article}
\usepackage{amssymb}
\usepackage{amsmath}
\usepackage[utf8]{inputenc} % Umlaute
\usepackage[ngerman]{babel} % Umlaute
\usepackage[T1]{fontenc}    % Umlaute
\usepackage[margin=2.5cm]{geometry}
\usepackage{booktabs}
\usepackage{lmodern}
\usepackage{titlesec}
\usepackage{longtable}
% Notwendig für Links im Text
\usepackage{hyperref}
%%\usepackage{svg}
% glossar, see http://en.wikibooks.org/wiki/LaTeX/Glossary
% muss NACH hyperref geladen werden, sonst funktionieren die Links nicht
\usepackage[toc]{glossaries}

% Kompatibilität
\ifx\pdftexversion\undefined
\usepackage[dvips]{graphicx}
\else
\usepackage[pdftex]{graphicx}
\DeclareGraphicsRule{*}{mps}{*}{}
\fi
\setlength{\parindent}{0pt}

%valle will Automaten verwenden
\usepackage{tikz}
\usetikzlibrary{automata, positioning, arrows}

%irgendwas mit section formatierung (titlesec package)
\titleformat{\paragraph}[hang]{\normalfont\normalsize\bfseries}{\theparagraph}{1em}{}
%%%%%%%%%%%%%%%%%%%%%%%%%%%%%%%%%%%%%%%%%%%%%%%%%%%%%%%%%%%%%%%%%%%%%%
% Variablen                                 						 %
%%%%%%%%%%%%%%%%%%%%%%%%%%%%%%%%%%%%%%%%%%%%%%%%%%%%%%%%%%%%%%%%%%%%%%
\newcommand{\authorName}{Tec O'Brain (Entwickler: David Höglinger, Jan Ettrich, Erwin Müller, Benedikt Rittner, Valentin Quapil)}
\newcommand{\auftraggeber}{Karlsruhe Institute of Technology (Teco)}
\newcommand{\auftragnehmer}{\authorName}
\newcommand{\projektName}{Validierung Earables}
\newcommand{\tags}{\authorName, Validierung, Tests, KIT, Informatik, PSE}
\newcommand{\documentVersion}{1.0}
\title{\projektName}
\date{\today}
\author{Tec O'Brain}

%%%%%%%%%%%%%%%%%%%%%%%%%%%%%%%%%%%%%%%%%%%%%%%%%%%%%%%%%%%%%%%%%%%%%%
% PDF Meta information                                 				 %
%%%%%%%%%%%%%%%%%%%%%%%%%%%%%%%%%%%%%%%%%%%%%%%%%%%%%%%%%%%%%%%%%%%%%%
\hypersetup{
  pdfauthor   = {\authorName},
  pdfkeywords = {\tags},
  pdftitle    = {\projektName)}
}

%%%%%%%%%%%%%%%%%%%%%%%%%%%%%%%%%%%%%%%%%%%%%%%%%%%%%%%%%%%%%%%%%%%%%%
% Create a shorter version for tables. DO NOT CHANGE               	 %
%%%%%%%%%%%%%%%%%%%%%%%%%%%%%%%%%%%%%%%%%%%%%%%%%%%%%%%%%%%%%%%%%%%%%%
\newcommand\addrow[2]{#1 &#2\\ }

\newcommand\addheading[2]{#1 &#2\\ \hline}
\newcommand\tabularhead{\begin{tabular}{lp{13cm}}
\hline
}

\newcommand\addmulrow[2]{ \begin{minipage}[t][][t]{2.5cm}#1\end{minipage}%
   &\begin{minipage}[t][][t]{8cm}
    \begin{enumerate} #2   \end{enumerate}
    \end{minipage}\\ }

\newenvironment{usecase}{\tabularhead}
{\hline\end{tabular}}

\usepackage{microtype}
%%%%%%%%%%%%%%%%%%%%%%%%%%%%%%%%%%%%%%%%%%%%%%%%%%%%%%%%%%%%%%%%%%%%%%
% GLOSSARY ENTRIES                 	                              	 %
%%%%%%%%%%%%%%%%%%%%%%%%%%%%%%%%%%%%%%%%%%%%%%%%%%%%%%%%%%%%%%%%%%%%%%


%%%%%%%%%%%%%%%%%%%%%%%%%%%%%%%%%%%%%%%%%%%%%%%%%%%%%%%%%%%%%%%%%%%%%%
% THE DOCUMENT BEGINS             	                              	 %
%%%%%%%%%%%%%%%%%%%%%%%%%%%%%%%%%%%%%%%%%%%%%%%%%%%%%%%%%%%%%%%%%%%%%%
\begin{document}
\pagenumbering{roman}
\begin{titlepage}
\maketitle
\thispagestyle{empty} % no page number

\begin{verbatim}












\end{verbatim}


  \begin{tabular}[t]{p{4 cm}p{8 cm}}
	Projekt:       & \projektName \\[1.2ex]
	Auftraggeber:  & \auftraggeber\\[1.2ex]
	Auftragnehmer: & \auftragnehmer\\[1.2ex]
  \end{tabular}


\begin{tabular}[t]{|p{4 cm}|p{8 cm}|}
\hline
\textbf{Datum} & \textbf{Autor(en)} \\
\hline
\hline
\today & \authorName \\
\hline
\end{tabular}
\end{titlepage}
         % Deckblatt.tex laden und einfügen
\setcounter{page}{2}
\tableofcontents          % Inhaltsverzeichnis ausgeben
\clearpage
\pagenumbering{arabic}
%%%%%%%%%%%%%%%%%%%%%%%%%%%%%%%%%%%%%%% CONTENT %%%%%%%%%%%%%%%%%%%%%%%%%%%%%%%%%%%%%%%%%%%%%%%

\section{Einleitung}
Das nachfolgende Dokument spiegelt den Ablauf unserer Qualitätssicherungsphase wieder.


\section{Testszenarien}

%Table template:
\iffalse
\begin{tabular}{ |p{1.5cm} | p{12cm} | c| }
	\hline
	\textbf{Testfall} & \textbf{Beschreibung} & \textbf{Bestanden}\\
	\hline
	/T070/ & Der Nutzer startet die App. & OK\\
	\hline	
\end{tabular}
\fi
Die Testszenarien setzen sich aus den im Pflichtenheft genannten Testfällen (T) zusammen.


\subsection{Laufmodus}
Änderungen: Keine
\\
\\
\begin{tabular}{ |p{1.5cm} | p{12cm} | c| }
	\hline
	\textbf{Testfall} & \textbf{Beschreibung} & \textbf{Bestanden}\\
	\hline
	/T070/ & Der Nutzer startet die App. & OK\\
	\hline
	& Der Nutzer verbindet in der App die Earables, über BLE mit seinem Smartphone. & OK\\
	\hline
	& Nach dem Startvorgang der App wird in den Modus \glqq Laufmodus\grqq{} gewechselt. & OK\\
	\hline
	/T104/ & Falls es schon einen Laufvorgang zuvor gab, werden nun die Anzahl der zurückgelegten Schritte und die zurückgelegte Distanz angezeigt.& OK\\
	\hline
	& Die Earables werden korrekt am Ohr des Nutzers angebracht. & OK\\
	\hline
	& Der Nutzer startet den Laufmodus. & OK\\
	\hline
	& Die App zeigt dem Nutzer den Status \glqq stehend\grqq{} an. & OK\\
	\hline
	& Die App zeigt dem Nutzer die aktuelle Schrittfrequenz und die Anzahl der bisher. & OK\\
	\hline
	& Sobald der Nutzer anfängt zu gehen zeigt die App \glqq gehend\grqq{} an. & OK\\
	\hline
	& Sobald der Nutzer wieder still steht ändert sich der Zustand wieder zurück zu "stehend". & OK\\
	\hline	
\end{tabular}


\subsection{Zählmodus}
Änderungen: Keine
\\
\\
\begin{tabular}{ |p{1.5cm} | p{12cm} | c| }
	\hline
	\textbf{Testfall} & \textbf{Beschreibung} & \textbf{Bestanden}\\
	\hline
	/T070/ & Der Nutzer startet die App. & OK\\
	\hline
	& Der Nutzer verbindet in der App die Earables, über BLE mit seinem Smartphone. & OK\\
	\hline
	/T090/ & Nach dem Startvorgang der App wechselt der Nutzer in den Modus \glqq Zählmodus\grqq . & OK\\
	\hline
	& Die Earables werden korrekt am Ohr des Nutzers angebracht. & OK \\
	\hline
	& Der Nutzer wählt eine verfügbare Übung aus. & OK \\
	\hline
	& Der Nutzer startet den gewählten Vorgang. & OK \\
	\hline
	& Der Nutzer führt die Übung 15 (natürliche Zahl) mal aus. & OK \\
	\hline
	& Der Nutzer stoppt den laufenden Vorgang. & OK \\
	\hline
	/T200/ & Die App zeigt 15 an. & OK \\
	\hline
\end{tabular}

\subsection{Start/Stop Musikmodus}
Änderungen: Die Funktion wurde aus dem Laufmodus in den Modus \glqq Musik Modus\grqq{} umbenannt.
\\
\\
\begin{tabular}{ |p{1.5cm} | p{12cm} | c| }
	\hline
	\textbf{Testfall} & \textbf{Beschreibung} & \textbf{Bestanden}\\
	\hline
	/T070/ & Der Nutzer startet die App. & OK\\
	\hline
	& Der Nutzer verbindet in der App die Earables, über BLE mit seinem Smartphone. & OK\\
	\hline
	& Der Nutzer spielt Musik mit der vorinstallierten Musik-App ab. & OK\\
	\hline
	& Nach dem Startvorgang der App wechselt der Nutzer in den Modus \glqq Music Mode\grqq . & OK\\
	\hline
	& Die Earables werden korrekt am Ohr des Nutzers angebracht. & OK\\
	\hline
	& Der Nutzer beginnt zu gehen. & OK \\
	\hline
	& Der Nutzer startet den gewählten Vorgang. & OK \\
	\hline
	& Der Nutzer hört auf zu gehen. & OK \\
	\hline
	/T210/ & Die Musik wird pausiert. & OK \\
	\hline
	& Der Nutzer geht weiter. & OK \\
	\hline
	/T210/ & Die Musik startet automatisch wieder. & OK \\
	\hline
\end{tabular}


\subsection{Lauschen\&Agieren}
Änderungen:
\\
\\
\begin{tabular}{ |p{1.5cm} | p{12cm} | c| }
	\hline
	\textbf{Testfall} & \textbf{Beschreibung} & \textbf{Bestanden}\\
	\hline
	/T070/ & Der Nutzer startet die App. & OK\\
	\hline
	& Der Nutzer verbindet in der App die Earables, über BLE mit seinem Smartphone. & OK \\
	\hline
	/F090/ & Nach dem Startvorgang der App wechselt der Nutzer in den Modus \glqq Lauschen\&Agieren\grqq . & OK \\
	\hline
	& Die Earables werden korrekt am Ohr des Nutzers angebracht. & OK \\
	\hline
	/T221/ & Der Nutzer stellt sich ein Training aus den verfügbaren Übungen zusammen. & OK \\
	\hline
	& Der Nutzer startet den gewählten Vorgang. & OK \\
	\hline
	/T222/ & Dem Nutzer wird per Text-To-Speech die aktuelle Übung angesagt. & OK \\
	\hline
	& Der Nutzer führt die Übung aus. & OK \\
	\hline
	& Die letzten beiden Schritte werden so lange wiederholt, bis alle ausgewählten Übungen erledigt sind. & OK \\
	\hline
	& Der Vorgang wird automatisch beendet. & OK \\
	\hline
	/T223/ & Die App zeigt die benötigte Zeit für den Vorgang an. & OK \\
	\hline
\end{tabular}

\subsection{Einstellungen}
Änderungen: Die Standardsprache ist Englisch, daher wurde /T260/ auf Deutsch geändert.
\\
\\
\begin{tabular}{ |p{1.5cm} | p{12cm} | c| }
	\hline
	\textbf{Testfall} & \textbf{Beschreibung} & \textbf{Bestanden}\\
	\hline
	/T070/ & Der Nutzer startet die App. & OK\\
	\hline
	& Der Nutzer verbindet in der App die Earables, über BLE mit seinem Smartphone. & OK \\
	\hline
	& Nach dem Startvorgang der App wechselt der Nutzer in die Ansicht \glqq Einstellungen\grqq . & OK\\
	\hline
	/T250/ & Der Nutzer ändert seinen Namen. & OK \\
	\hline
	/T260/ & Der Nutzer ändert die Sprache auf Deutsch & OK \\
	\hline
	/T270/ & Dem Nutzer löscht die gespeicherten Vorgangsdaten. & OK \\
	\hline
	/T280/ & Dem Nutzer ändert die Samplingrate & OK \\
	\hline
	/T285/ & Der Nutzer ändert seine Schrittlänge. & OK \\
	\hline
	& Der Nutzer klickt auf speichern. & OK \\
	\hline
	& Der Nutzer beendet die App. & OK \\
	\hline
	/T070/ & Der Nutzer startet die App. & OK \\
	\hline
	& Der Nutzer wechselt in die Ansicht \glqq Einstellungen\grqq . & OK \\
	\hline
	& Die Einstellungen sind exakt so, wie sie vorher eingestellt wurden. & OK \\
	\hline
\end{tabular}


\subsection{Erstnutzung}
Änderungen: Der Punkt \glqq Der Nutzer klickt auf SAVE\grqq{} wurde hinzugefügt. Außerdem wurde die Sprache von Englisch auf Deutsch geändert.
\\
\\
\begin{tabular}{ |p{1.5cm} | p{12cm} | c| }
	\hline
	\textbf{Testfall} & \textbf{Beschreibung} & \textbf{Bestanden}\\
	\hline
	/T070/ & Der Nutzer startet die App. & OK\\
	\hline
	/T290/ & Der Nutzer wird aufgefordert seinen Namen und seine Schrittlänge zu setzen. & OK \\
	\hline
	& Der Nutzer wählt im Hamburgermenü den Reiter Einstellungen aus. & OK \\
	\hline
	/T260/ & Der Nutzer ändert die Sprache auf Deutsch. & OK \\
	\hline
	& Der Nutzer klickt auf \glqq SAVE\grqq . & OK \\
	\hline
	& Danach wechselt der Nutzer in den \glqq Laufmodus\grqq . & OK \\
	\hline
	& Der Nutzer schließt die App. & OK \\
	\hline
	/T070/ & Der Nutzer startet die App erneut und befindet sich nun im \glqq Laufmodus\grqq . & OK \\
	\hline	
	& Sein Name, seine Schrittlänge und die in-App Sprache wurden gespeichert. & OK \\
	\hline
\end{tabular}



\subsection{Vorgangsdaten importieren/exportieren}
Änderungen: \glqq Import/Export\grqq{} wurde in \glqq Dateimanagement\grqq umbenannt.
\\
\\
\begin{tabular}{ |p{1.5cm} | p{12cm} | c| }
	\hline
	\textbf{Testfall} & \textbf{Beschreibung} & \textbf{Bestanden}\\
	\hline
	/T070/ & Der Nutzer startet die App. & OK\\
	\hline
	& Der Nutzer wählt über das Hamburgermenü den Reiter \glqq Dateimanagement\grqq . & OK \\
	\hline
	& Der Nutzer exportiert seine Daten. & OK\\
	\hline
	/T270/ & Der Nutzer löscht seine Daten. & OK\\
	\hline
	/T300/ & Der Nutzer importiert seine Daten aus einer CSV-Datei. & OK\\
	\hline
	& Die Daten sind alle wieder vorhanden so als ob der Nutzer sie nie gelöscht hätte. & OK\\
	\hline
\end{tabular}

\subsection{Trainingsdaten anschauen}
Änderungen: -
\\
\\
\begin{tabular}{ |p{1.5cm} | p{12cm} | c| }
	\hline
	\textbf{Testfall} & \textbf{Beschreibung} & \textbf{Bestanden}\\
	\hline
	/T070/ & Der Nutzer startet die App. & OK\\
	\hline
	& Der Nutzer wechselt über das Hamburgermenü in die Datenübersicht. & OK \\
	\hline
	/T320/ & Dem Nutzer werden die letzten Trainingsdaten der letzten 30 Vorgangstage angezeigt. & OK \\
	\hline
\end{tabular}

\subsection{Pop-up}
Änderungen: -
\\
\\
\begin{tabular}{ |p{1.5cm} | p{12cm} | c| }
	\hline
	\textbf{Testfall} & \textbf{Beschreibung} & \textbf{Bestanden}\\
	\hline
	/T070/ & Der Nutzer startet die App. & OK\\
	\hline
	& Das Pop-up Fenster erscheint. & OK \\
	\hline
	/T160/ & Der Nutzer klickt das Pop-up weg. & OK \\
	\hline
	& Der Nutzer versucht einen Laufvorgang zu starten. & OK \\
	\hline
	& Das Pop-up erscheint. & OK \\
	\hline
	& Der Nutzer stellt eine Verbindung zu den Earables her. & OK \\
	\hline
	& Der Nutzer entfernt sich mit seinem Smartphone von de Earables. & OK \\
	\hline
	/T150/ & Das Smartphone verliert die Bluetooth Verbindung zu den Earables und das Pop-up erscheint. & OK \\
	\hline
\end{tabular}



\section{Test-Coverage}

\section{Regressionstests}

\section{Beschreibung Fehler}

\section{Anhang}
\subsection{Unit Test Reports}
\subsubsection{Bibliothek}
\subsubsection{App}

\subsection{Links}
\subsubsection{Bibliothek}
NuGet Package: \url{https://www.nuget.org/packages/EarablesBLE}\\
GitHub Repository: \url{https://github.com/vlle1/lib-earablesKIT}
\subsubsection{App}
GitHub Repository: \url{https://github.com/vlle1/earablesKIT}
%%%%%%%%%%%%%%%%%%%%%%%%%%%%%%%%%%%%%%% END CONTENT %%%%%%%%%%%%%%%%%%%%%%%%%%%%%%%%%%%%%%%%%%%


\printglossaries
\stepcounter{section}


\end{document}