\documentclass[a4paper,12pt]{article}
\usepackage{amssymb}
\usepackage{amsmath}
\usepackage[utf8]{inputenc} % Umlaute
\usepackage[ngerman]{babel} % Umlaute
\usepackage[T1]{fontenc}    % Umlaute
\usepackage[margin=2.5cm]{geometry}
\usepackage{booktabs}
\usepackage{lmodern}
\usepackage{titlesec}
\usepackage{longtable}
% Notwendig für Links im Text
\usepackage{hyperref}
%%\usepackage{svg}
% glossar, see http://en.wikibooks.org/wiki/LaTeX/Glossary
% muss NACH hyperref geladen werden, sonst funktionieren die Links nicht
\usepackage[toc]{glossaries}

% Kompatibilität
\ifx\pdftexversion\undefined
\usepackage[dvips]{graphicx}
\else
\usepackage[pdftex]{graphicx}
\DeclareGraphicsRule{*}{mps}{*}{}
\fi
\setlength{\parindent}{0pt}

%valle will Automaten verwenden
\usepackage{tikz}
\usetikzlibrary{automata, positioning, arrows}

%irgendwas mit section formatierung (titlesec package)
\titleformat{\paragraph}[hang]{\normalfont\normalsize\bfseries}{\theparagraph}{1em}{}
%%%%%%%%%%%%%%%%%%%%%%%%%%%%%%%%%%%%%%%%%%%%%%%%%%%%%%%%%%%%%%%%%%%%%%
% Variablen                                 						 %
%%%%%%%%%%%%%%%%%%%%%%%%%%%%%%%%%%%%%%%%%%%%%%%%%%%%%%%%%%%%%%%%%%%%%%
\newcommand{\authorName}{Tec O'Brain (Entwickler: David Höglinger, Jan Ettrich, Erwin Müller, Benedikt Rittner, Valentin Quapil)}
\newcommand{\auftraggeber}{Karlsruhe Institute of Technology (Teco)}
\newcommand{\auftragnehmer}{\authorName}
\newcommand{\projektName}{Validierung Earables}
\newcommand{\tags}{\authorName, Validierung, Tests, KIT, Informatik, PSE}
\newcommand{\documentVersion}{1.0}
\title{\projektName}
\date{\today}
\author{Tec O'Brain}

%%%%%%%%%%%%%%%%%%%%%%%%%%%%%%%%%%%%%%%%%%%%%%%%%%%%%%%%%%%%%%%%%%%%%%
% PDF Meta information                                 				 %
%%%%%%%%%%%%%%%%%%%%%%%%%%%%%%%%%%%%%%%%%%%%%%%%%%%%%%%%%%%%%%%%%%%%%%
\hypersetup{
  pdfauthor   = {\authorName},
  pdfkeywords = {\tags},
  pdftitle    = {\projektName)}
}

%%%%%%%%%%%%%%%%%%%%%%%%%%%%%%%%%%%%%%%%%%%%%%%%%%%%%%%%%%%%%%%%%%%%%%
% Create a shorter version for tables. DO NOT CHANGE               	 %
%%%%%%%%%%%%%%%%%%%%%%%%%%%%%%%%%%%%%%%%%%%%%%%%%%%%%%%%%%%%%%%%%%%%%%
\newcommand\addrow[2]{#1 &#2\\ }

\newcommand\addheading[2]{#1 &#2\\ \hline}
\newcommand\tabularhead{\begin{tabular}{lp{13cm}}
\hline
}

\newcommand\addmulrow[2]{ \begin{minipage}[t][][t]{2.5cm}#1\end{minipage}%
   &\begin{minipage}[t][][t]{8cm}
    \begin{enumerate} #2   \end{enumerate}
    \end{minipage}\\ }

\newenvironment{usecase}{\tabularhead}
{\hline\end{tabular}}

\usepackage{microtype}
%%%%%%%%%%%%%%%%%%%%%%%%%%%%%%%%%%%%%%%%%%%%%%%%%%%%%%%%%%%%%%%%%%%%%%
% GLOSSARY ENTRIES                 	                              	 %
%%%%%%%%%%%%%%%%%%%%%%%%%%%%%%%%%%%%%%%%%%%%%%%%%%%%%%%%%%%%%%%%%%%%%%


%%%%%%%%%%%%%%%%%%%%%%%%%%%%%%%%%%%%%%%%%%%%%%%%%%%%%%%%%%%%%%%%%%%%%%
% THE DOCUMENT BEGINS             	                              	 %
%%%%%%%%%%%%%%%%%%%%%%%%%%%%%%%%%%%%%%%%%%%%%%%%%%%%%%%%%%%%%%%%%%%%%%
\begin{document}
\pagenumbering{roman}
\begin{titlepage}
\maketitle
\thispagestyle{empty} % no page number

\begin{verbatim}












\end{verbatim}


  \begin{tabular}[t]{p{4 cm}p{8 cm}}
	Projekt:       & \projektName \\[1.2ex]
	Auftraggeber:  & \auftraggeber\\[1.2ex]
	Auftragnehmer: & \auftragnehmer\\[1.2ex]
  \end{tabular}


\begin{tabular}[t]{|p{4 cm}|p{8 cm}|}
\hline
\textbf{Datum} & \textbf{Autor(en)} \\
\hline
\hline
\today & \authorName \\
\hline
\end{tabular}
\end{titlepage}
         % Deckblatt.tex laden und einfügen
\setcounter{page}{2}
\tableofcontents          % Inhaltsverzeichnis ausgeben
\clearpage
\pagenumbering{arabic}
%%%%%%%%%%%%%%%%%%%%%%%%%%%%%%%%%%%%%%% CONTENT %%%%%%%%%%%%%%%%%%%%%%%%%%%%%%%%%%%%%%%%%%%%%%%

\section{Einleitung}
Das nachfolgende Dokument spiegelt den Ablauf unserer Qualitätssicherungsphase wieder.


\section{Testszenarien}

%Table template:
\iffalse
\begin{tabular}{ |p{1.5cm} | p{12cm} | c| }
	\hline
	\textbf{Testfall} & \textbf{Beschreibung} & \textbf{Bestanden}\\
	\hline
	/T070/ & Der Nutzer startet die App. & OK\\
	\hline	
\end{tabular}
\fi
Die Testszenarien setzen sich aus den im Pflichtenheft genannten Testfällen (T) zusammen.


\subsection{Laufmodus}
Änderungen: Keine
\\
\\
\begin{tabular}{ |p{1.5cm} | p{12cm} | c| }
	\hline
	\textbf{Testfall} & \textbf{Beschreibung} & \textbf{Bestanden}\\
	\hline
	/T070/ & Der Nutzer startet die App. & OK\\
	\hline
	& Der Nutzer verbindet in der App die Earables, über BLE mit seinem Smartphone. & OK\\
	\hline
	& Nach dem Startvorgang der App wird in den Modus \glqq Laufmodus\grqq{} gewechselt. & OK\\
	\hline
	/T104/ & Falls es schon einen Laufvorgang zuvor gab, werden nun die Anzahl der zurückgelegten Schritte und die zurückgelegte Distanz angezeigt.& OK\\
	\hline
	& Die Earables werden korrekt am Ohr des Nutzers angebracht. & OK\\
	\hline
	& Der Nutzer startet den Laufmodus. & OK\\
	\hline
	& Die App zeigt dem Nutzer den Status \glqq stehend\grqq{} an. & OK\\
	\hline
	& Die App zeigt dem Nutzer die aktuelle Schrittfrequenz und die Anzahl der bisher. & OK\\
	\hline
	& Sobald der Nutzer anfängt zu gehen zeigt die App \glqq gehend\grqq{} an. & OK\\
	\hline
	& Sobald der Nutzer wieder still steht ändert sich der Zustand wieder zurück zu "stehend". & OK\\
	\hline	
\end{tabular}


\subsection{Zählmodus}
Änderungen: Keine
\\
\\
\begin{tabular}{ |p{1.5cm} | p{12cm} | c| }
	\hline
	\textbf{Testfall} & \textbf{Beschreibung} & \textbf{Bestanden}\\
	\hline
	/T070/ & Der Nutzer startet die App. & OK\\
	\hline
	& Der Nutzer verbindet in der App die Earables, über BLE mit seinem Smartphone. & OK\\
	\hline
	/T090/ & Nach dem Startvorgang der App wechselt der Nutzer in den Modus \glqq Zählmodus\grqq . & OK\\
	\hline
	& Die Earables werden korrekt am Ohr des Nutzers angebracht. & OK \\
	\hline
	& Der Nutzer wählt eine verfügbare Übung aus. & OK \\
	\hline
	& Der Nutzer startet den gewählten Vorgang. & OK \\
	\hline
	& Der Nutzer führt die Übung 15 (natürliche Zahl) mal aus. & OK \\
	\hline
	& Der Nutzer stoppt den laufenden Vorgang. & OK \\
	\hline
	/T200/ & Die App zeigt 15 an. & OK \\
	\hline
\end{tabular}

\subsection{Start/Stop Musikmodus}
Änderungen: Die Funktion wurde aus dem Laufmodus in den Modus \glqq Musik Modus\grqq{} umbenannt.
\\
\\
\begin{tabular}{ |p{1.5cm} | p{12cm} | c| }
	\hline
	\textbf{Testfall} & \textbf{Beschreibung} & \textbf{Bestanden}\\
	\hline
	/T070/ & Der Nutzer startet die App. & OK\\
	\hline
	& Der Nutzer verbindet in der App die Earables, über BLE mit seinem Smartphone. & OK\\
	\hline
	& Der Nutzer spielt Musik mit der vorinstallierten Musik-App ab. & OK\\
	\hline
	& Nach dem Startvorgang der App wechselt der Nutzer in den Modus \glqq Music Mode\grqq . & OK\\
	\hline
	& Die Earables werden korrekt am Ohr des Nutzers angebracht. & OK\\
	\hline
	& Der Nutzer beginnt zu gehen. & OK \\
	\hline
	& Der Nutzer startet den gewählten Vorgang. & OK \\
	\hline
	& Der Nutzer hört auf zu gehen. & OK \\
	\hline
	/T210/ & Die Musik wird pausiert. & OK \\
	\hline
	& Der Nutzer geht weiter. & OK \\
	\hline
	/T210/ & Die Musik startet automatisch wieder. & OK \\
	\hline
\end{tabular}


\subsection{Lauschen\&Agieren}
Änderungen:
\\
\\
\begin{tabular}{ |p{1.5cm} | p{12cm} | c| }
	\hline
	\textbf{Testfall} & \textbf{Beschreibung} & \textbf{Bestanden}\\
	\hline
	/T070/ & Der Nutzer startet die App. & OK\\
	\hline
	& Der Nutzer verbindet in der App die Earables, über BLE mit seinem Smartphone. & OK \\
	\hline
	/F090/ & Nach dem Startvorgang der App wechselt der Nutzer in den Modus \glqq Lauschen\&Agieren\grqq . & OK \\
	\hline
	& Die Earables werden korrekt am Ohr des Nutzers angebracht. & OK \\
	\hline
	/T221/ & Der Nutzer stellt sich ein Training aus den verfügbaren Übungen zusammen. & OK \\
	\hline
	& Der Nutzer startet den gewählten Vorgang. & OK \\
	\hline
	/T222/ & Dem Nutzer wird per Text-To-Speech die aktuelle Übung angesagt. & OK \\
	\hline
	& Der Nutzer führt die Übung aus. & OK \\
	\hline
	& Die letzten beiden Schritte werden so lange wiederholt, bis alle ausgewählten Übungen erledigt sind. & OK \\
	\hline
	& Der Vorgang wird automatisch beendet. & OK \\
	\hline
	/T223/ & Die App zeigt die benötigte Zeit für den Vorgang an. & OK \\
	\hline
\end{tabular}

\subsection{Einstellungen}
Änderungen: Die Standardsprache ist Englisch, daher wurde /T260/ auf Deutsch geändert.
\\
\\
\begin{tabular}{ |p{1.5cm} | p{12cm} | c| }
	\hline
	\textbf{Testfall} & \textbf{Beschreibung} & \textbf{Bestanden}\\
	\hline
	/T070/ & Der Nutzer startet die App. & OK\\
	\hline
	& Der Nutzer verbindet in der App die Earables, über BLE mit seinem Smartphone. & OK \\
	\hline
	& Nach dem Startvorgang der App wechselt der Nutzer in die Ansicht \glqq Einstellungen\grqq . & OK\\
	\hline
	/T250/ & Der Nutzer ändert seinen Namen. & OK \\
	\hline
	/T260/ & Der Nutzer ändert die Sprache auf Deutsch & OK \\
	\hline
	/T270/ & Dem Nutzer löscht die gespeicherten Vorgangsdaten. & OK \\
	\hline
	/T280/ & Dem Nutzer ändert die Samplingrate & OK \\
	\hline
	/T285/ & Der Nutzer ändert seine Schrittlänge. & OK \\
	\hline
	& Der Nutzer klickt auf speichern. & OK \\
	\hline
	& Der Nutzer beendet die App. & OK \\
	\hline
	/T070/ & Der Nutzer startet die App. & OK \\
	\hline
	& Der Nutzer wechselt in die Ansicht \glqq Einstellungen\grqq . & OK \\
	\hline
	& Die Einstellungen sind exakt so, wie sie vorher eingestellt wurden. & OK \\
	\hline
\end{tabular}


\subsection{Erstnutzung}
Änderungen: Der Punkt \glqq Der Nutzer klickt auf SAVE\grqq{} wurde hinzugefügt. Außerdem wurde die Sprache von Englisch auf Deutsch geändert.
\\
\\
\begin{tabular}{ |p{1.5cm} | p{12cm} | c| }
	\hline
	\textbf{Testfall} & \textbf{Beschreibung} & \textbf{Bestanden}\\
	\hline
	/T070/ & Der Nutzer startet die App. & OK\\
	\hline
	/T290/ & Der Nutzer wird aufgefordert seinen Namen und seine Schrittlänge zu setzen. & OK \\
	\hline
	& Der Nutzer wählt im Hamburgermenü den Reiter Einstellungen aus. & OK \\
	\hline
	/T260/ & Der Nutzer ändert die Sprache auf Deutsch. & OK \\
	\hline
	& Der Nutzer klickt auf \glqq SAVE\grqq . & OK \\
	\hline
	& Danach wechselt der Nutzer in den \glqq Laufmodus\grqq . & OK \\
	\hline
	& Der Nutzer schließt die App. & OK \\
	\hline
	/T070/ & Der Nutzer startet die App erneut und befindet sich nun im \glqq Laufmodus\grqq . & OK \\
	\hline	
	& Sein Name, seine Schrittlänge und die in-App Sprache wurden gespeichert. & OK \\
	\hline
\end{tabular}



\subsection{Vorgangsdaten importieren/exportieren}
Änderungen: \glqq Import/Export\grqq{} wurde in \glqq Dateimanagement\grqq umbenannt.
\\
\\
\begin{tabular}{ |p{1.5cm} | p{12cm} | c| }
	\hline
	\textbf{Testfall} & \textbf{Beschreibung} & \textbf{Bestanden}\\
	\hline
	/T070/ & Der Nutzer startet die App. & OK\\
	\hline
	& Der Nutzer wählt über das Hamburgermenü den Reiter \glqq Dateimanagement\grqq . & OK \\
	\hline
	& Der Nutzer exportiert seine Daten. & OK\\
	\hline
	/T270/ & Der Nutzer löscht seine Daten. & OK\\
	\hline
	/T300/ & Der Nutzer importiert seine Daten aus einer CSV-Datei. & OK\\
	\hline
	& Die Daten sind alle wieder vorhanden so als ob der Nutzer sie nie gelöscht hätte. & OK\\
	\hline
\end{tabular}

\subsection{Trainingsdaten anschauen}
Änderungen: -
\\
\\
\begin{tabular}{ |p{1.5cm} | p{12cm} | c| }
	\hline
	\textbf{Testfall} & \textbf{Beschreibung} & \textbf{Bestanden}\\
	\hline
	/T070/ & Der Nutzer startet die App. & OK\\
	\hline
	& Der Nutzer wechselt über das Hamburgermenü in die Datenübersicht. & OK \\
	\hline
	/T320/ & Dem Nutzer werden die letzten Trainingsdaten der letzten 30 Vorgangstage angezeigt. & OK \\
	\hline
\end{tabular}

\subsection{Pop-up}
Änderungen: -
\\
\\
\begin{tabular}{ |p{1.5cm} | p{12cm} | c| }
	\hline
	\textbf{Testfall} & \textbf{Beschreibung} & \textbf{Bestanden}\\
	\hline
	/T070/ & Der Nutzer startet die App. & OK\\
	\hline
	& Das Pop-up Fenster erscheint. & OK \\
	\hline
	/T160/ & Der Nutzer klickt das Pop-up weg. & OK \\
	\hline
	& Der Nutzer versucht einen Laufvorgang zu starten. & OK \\
	\hline
	& Das Pop-up erscheint. & OK \\
	\hline
	& Der Nutzer stellt eine Verbindung zu den Earables her. & OK \\
	\hline
	& Der Nutzer entfernt sich mit seinem Smartphone von de Earables. & OK \\
	\hline
	/T150/ & Das Smartphone verliert die Bluetooth Verbindung zu den Earables und das Pop-up erscheint. & OK \\
	\hline
\end{tabular}



\section{Test-Coverage}

\section{Bib Test App}

%%\subsection{Einleitung}

%%Da wir für die Bluetoothverbindung in unserer Bibliothek das NuGet "Bluetooth LE plugin for Xamarin" benutzen erschwert dies das Testen mit XUnit tests, da die Bibliothek die Bits und Bytes der Earables empfängt und schreibt und ohne eine Verbindung zu den Earables funktioniert das nicht so gut. Damit wir nun nicht die komplette Bluetooth Verindung mocken müssen, haben wir uns dazu entschieden eine Test App zu bauen, die unsere Bibliothek Testet.

%%\subsection{Die Bib Test App}
Wir haben eine eigene App gebaut um die Funktionalität unsere Bibliothek zu testen. So sind wir in der Lage, mit kleine Szenarien, die nur unsere Bibliothek betreffen, zielgerichtete Tests durchzuführen.
Mit der Test App werden alle Funktionalitäten getestet, die die Bibliothek bereitstellt. Diese sind das Suchen von Geräten über Bluetooth, das Verbinden mit Bluetooth Geräten (speziell mit Earables), das Trennen von Bluetooth Geräten, das Setzen des LPF für das Gyroscope, das Auslesen des LPF für das Gyroscope, das Setzen des LPF für den Accelerometer, das Auslesen des LPF für den Accelerometer, das Starten des samplings, das Stoppen des samplings, das Setzen der Samplingrate, das Auslesen der Samplingrate, das Setzen der Range des Gyroscopes, das Auslesen des Scalefactors für das Gyroscope (hängt von der Range ab und wird für die Erzeugung der Rohdaten benötigt), das Setzen der Range des Accelerometers, das Auslesen des Scalefactors für den Accelerometer (hängt von der Range ab und wird für die Erzeugung der Rohdaten benötigt), das Anzeigen der Battery Voltage und das Ermitteln des Verbindungdsstatuses.

\subsection{Test Szenarien}

\subsubsection{Verbinden mit den Earables}
\begin{tabular}{ | p{12cm} | c| }
	\hline
	\textbf{Beschreibung} & \textbf{Bestanden}\\
	\hline
	Der Nutzer startet die App. & OK\\
	\hline
	Der Nutzer drückt den Button \glqq{}Get  connection status\grqq{}. & OK\\
	\hline
	Es erscheint ein Pop-up welches als connection status \glqq{}getrennt\grqq{} anzeigt. & OK\\
	\hline
	Der Nutzer klickt das Pop-up weg. & OK\\
	\hline
	Der Nutzer drückt den Button \glqq{}Search\grqq{}. & OK \\
	\hline
	Der Nutzer wählt seine Earables, als zu verbindendes Device, aus. & OK \\
	\hline
	Es erscheint ein Pop-up, das dem Nutzer anzeigt, dass er nun, mit den Earables verbund ist. & OK \\
	\hline
	Der Nutzer klickt das Pop-up weg. & OK \\
	\hline
	Der Nutzer drückt den Button \glqq{}Get  connection status\grqq{}. & OK\\
	\hline
	Es erscheint ein Pop-up welches als connection status \glqq{}verbunden\grqq{} anzeigt. & OK\\
	\hline
	Der Nutzer klickt das Pop-up weg. & OK \\
	\hline
	Der Nutzer drückt den Button \glqq{}Disconnect\grqq{}. & OK\\
	\hline
	Es erscheint ein Pop-up welches als connection status \glqq{}getrennt\grqq{} anzeigt. & OK\\
	\hline
	Der Nutzer klickt das Pop-up weg. & OK\\
	\hline
	Der Nutzer drückt den Button \glqq{}Get  connection status\grqq{}. & OK\\
	\hline
	Es erscheint ein Pop-up welches als connection status \glqq{}getrennt\grqq{} anzeigt. & OK\\
	\hline
	Der Nutzer klickt das Pop-up weg. & OK\\
	\hline
\end{tabular}

Hinweis: Für die nachfolgenden Tests wird angenommen, dass die App bereits gestartet ist und eine Verbindung zu den Earable besteht.

\subsubsection{Sampling starten}
\begin{tabular}{ | p{12cm} | c| }
	\hline
	\textbf{Beschreibung} & \textbf{Bestanden}\\
	\hline
	Der Nutzer drückt den Button \glqq{}Start Sampling\grqq{}. & OK\\
	\hline
	Es werden die Rohdaten auf dem Bildschirm angezeigt. & OK\\
	\hline
	Der Nutzer bewegt seinen Kopf und sieht wie sich die Rohdaten dementsprechend verändern. & OK\\
	\hline
	Der Nutzer drückt den Button \glqq{}Stop Sampling\grqq{}. & OK \\
	\hline
	Das Sampling wird gestoppt und die Rohdaten hören auf sich zu aktualisieren. & OK \\
	\hline
\end{tabular}

%%mit einer anderen app die daten aufzeichnen und vregleichen ob beide das gleiche anzeigen
\subsubsection{Setzen der LPF für das Gyroscope}
\begin{tabular}{ | p{12cm} | c| }
	\hline
	\textbf{Beschreibung} & \textbf{Bestanden}\\
	\hline
	Der Nutzer drückt den Button \glqq{}Get Gyro LPF\grqq{}. & OK\\
	\hline
	Es wird angezeigt dass der LPF für das Gyroscope 5Hz beträgt . & OK\\
	\hline
	Der Nutzer drückt den Button \glqq{}Set Gyro LPF\grqq{}. & OK\\
	\hline
	Der Nutzer wählt 41Hz als LPF aus. & OK \\
	\hline
	Es wird angezeigt, dass der LPF auf 41Hz gesetzt wird . & OK \\
	\hline
	Der Nutzer drückt den Button \glqq{}Get Gyro LPF\grqq{}. & OK\\
	\hline
	Es wird angezeigt dass der LPF für das Gyroscope 41Hz beträgt . & OK\\
	\hline
\end{tabular}

\subsubsection{Setzen der LPF für den Accelerometer}
\begin{tabular}{ | p{12cm} | c| }
	\hline
	\textbf{Beschreibung} & \textbf{Bestanden}\\
	\hline
	Der Nutzer drückt den Button \glqq{}Get Acc LPF\grqq{}. & OK\\
	\hline
	Es wird angezeigt dass der LPF für den Acc 5Hz beträgt . & OK\\
	\hline
	Der Nutzer drückt den Button \glqq{}Set Acc LPF\grqq{}. & OK\\
	\hline
	Der Nutzer wählt 184Hz als LPF aus. & OK \\
	\hline
	Es wird angezeigt, dass der LPF auf 184Hz gesetzt wird . & OK \\
	\hline
	Der Nutzer drückt den Button \glqq{}Get Acc LPF\grqq{}. & OK\\
	\hline
	Es wird angezeigt dass der LPF für den Accelerometer 41Hz beträgt . & OK\\
	\hline
\end{tabular}

\subsubsection{Samplingrate verändern}
\begin{tabular}{ | p{12cm} | c| }
	\hline
	\textbf{Beschreibung} & \textbf{Bestanden}\\
	\hline
	Der Nutzer drückt den Button \glqq{}Start Sampling\grqq{}. & OK\\
	\hline
	Es werden die Rohdaten auf dem Bildschirm angezeigt und der Nutzer sieht, dass sie sich 50 mal pro Sekunde aktualisieren. & OK\\
	\hline
	Der Nutzer bewegt seinen Kopf und sieht wie sich die Rohdaten dementsprechend verändern. & OK\\
	\hline
	Der Nutzer drückt den Button \glqq{}Stop Sampling\grqq{}. & OK \\
	\hline
	Das Sampling wird gestoppt und die Rohdaten hören auf sich zu aktualisieren. & OK \\
	\hline
	Der Nutzer drückt den Button \glqq{}Get Samplingrate\grqq{}. & OK\\
	\hline
	Es wird angezeigt dass die Samplingrate 50Hz beträgt. & OK\\
	\hline
	Der Nutzer drückt den Button \glqq{}SetSamplingrate\grqq{}. & OK\\
	\hline
	Der Nutzer gibt 1 als Samplingrate ein. & OK \\
	\hline
	Es wird angezeigt, dass die Samplingrate auf 1 gesetzt wird. & OK \\
	\hline
	Der Nutzer drückt den Button \glqq{}Start Sampling\grqq{}. & OK\\
	\hline
	Es werden die Rohdaten auf dem Bildschirm angezeigt und der Nutzer sieht, dass sie sich nur ein mal pro Sekunde aktualisieren. & OK\\
	\hline
	Der Nutzer bewegt seinen Kopf und sieht wie sich die Rohdaten dementsprechend verändern. & OK\\
	\hline
	Der Nutzer drückt den Button \glqq{}Stop Sampling\grqq{}. & OK \\
	\hline
	Das Sampling wird gestoppt und die Rohdaten hören auf sich zu aktualisieren. & OK \\
	\hline
\end{tabular}
\subsubsection{Gyro Range verändern}
\begin{tabular}{ | p{12cm} | c| }
	\hline
	\textbf{Beschreibung} & \textbf{Bestanden}\\
	\hline
	Der Nutzer drückt den Button \glqq{}Get Gyro Scalefactor\grqq{}. & OK\\
	\hline
	Es wird angezeigt, dass der Gyro Scale Factor 65,5 beträgt. & OK\\
	\hline
	Der Nutzer drückt den Button \glqq{}Start Sampling\grqq{}. & OK\\
	\hline
	Es werden die Rohdaten auf dem Bildschirm angezeigt und der Nutzer erkennt, dass die Range der Rotationen um die Achsen auf +- 500 deg/s beschrängt ist, wenn er seinen Kopf dreht. & OK\\
	\hline
	Der Nutzer drückt den Button \glqq{}Stop Sampling\grqq{}. & OK \\
	\hline
	Das Sampling wird gestoppt und die Rohdaten hören auf sich zu aktualisieren. & OK \\
	\hline
	Der Nutzer drückt den Button \glqq{}Set Gyro Range\grqq{}. & OK\\
	\hline
	Er wählt 250deg/s aus. & OK\\
	\hline
	Es wird angezeigt, dass die Gyro Range jetzt 250deg/s beträgt. & OK\\
	\hline
	Der Nutzer drückt den Button \glqq{}Get Gyro Scalefactor\grqq{}. & OK\\
	\hline
	Es wird angezeigt, dass der Gyro Scale Factor 131 beträgt. & OK\\
	\hline
	Der Nutzer drückt den Button \glqq{}Start Sampling\grqq{}. & OK\\
	\hline
	Es werden die Rohdaten auf dem Bildschirm angezeigt und der Nutzer erkennt, dass die Range der Rotationen um die Achsen auf +- 250 deg/s beschrängt ist, wenn er seinen Kopf dreht. & OK\\
	\hline
	Der Nutzer drückt den Button \glqq{}Stop Sampling\grqq{}. & OK \\
	\hline
\end{tabular}
\subsubsection{Acc Range verändern}
\begin{tabular}{ | p{12cm} | c| }
	\hline
	\textbf{Beschreibung} & \textbf{Bestanden}\\
	\hline
	Der Nutzer drückt den Button \glqq{}Get Acc Scalefactor\grqq{}. & OK\\
	\hline
	Es wird angezeigt, dass der Gyro Scale Factor 8192 beträgt. & OK\\
	\hline
	Der Nutzer drückt den Button \glqq{}Start Sampling\grqq{}. & OK\\
	\hline
	Es werden die Rohdaten auf dem Bildschirm angezeigt und der Nutzer erkennt, dass die Range der Beschleunigungen auf den Achsen  auf +-4 g beschränkt ist, wenn er sich bewegt. & OK\\
	\hline
	Der Nutzer drückt den Button \glqq{}Stop Sampling\grqq{}. & OK \\
	\hline
	Das Sampling wird gestoppt und die Rohdaten hören auf sich zu aktualisieren. & OK \\
	\hline
	Der Nutzer drückt den Button \glqq{}Set Acc Range\grqq{}. & OK\\
	\hline
	Er wählt 2g aus. & OK\\
	\hline
	Es wird angezeigt, dass die Gyro Range jetzt 2g beträgt. & OK\\
	\hline
	Der Nutzer drückt den Button \glqq{}Get Acc Scalefactor\grqq{}. & OK\\
	\hline
	Es wird angezeigt, dass der Acc Scale Factor 16384 beträgt. & OK\\
	\hline
	Der Nutzer drückt den Button \glqq{}Start Sampling\grqq{}. & OK\\
	\hline
	Es werden die Rohdaten auf dem Bildschirm angezeigt und der Nutzer erkennt, dass die Range der Beschleunigungen auf den Achsen  auf +-2g beschränkt ist, wenn er sich bewegt. & OK\\
	\hline
	Der Nutzer drückt den Button \glqq{}Stop Sampling\grqq{}. & OK \\
	\hline
\end{tabular}

\subsubsection{BatteryVoltage auslesen}
\begin{tabular}{ | p{12cm} | c| }
	\hline
	\textbf{Beschreibung} & \textbf{Bestanden}\\
	\hline
	Der Nutzer drückt den Button \glqq{}Get Batteryvoltage\grqq{}. & OK\\
	\hline
	Es wird die aktuelle Batteryvoltage angezeigt. & OK\\
	\hline
\end{tabular}

\subsubsection{Push button drücken}
\begin{tabular}{ | p{12cm} | c| }
	\hline
	\textbf{Beschreibung} & \textbf{Bestanden}\\
	\hline
	Der Nutzer drückt denden Knopf an den Earables. & OK\\
	\hline
	Auf der Console, im Debug Modus, wird eine Zeile gedruckt, die dem Nutzer mitteilt, dass der Knopf an den Earables gedrückt wurde. & OK\\
	\hline
\end{tabular}

\subsubsection{Bluetooth status auslesen}
\begin{tabular}{ | p{12cm} | c| }
	\hline
	\textbf{Beschreibung} & \textbf{Bestanden}\\
	\hline
	Der Nutzer drückt den Button \glqq{}Get Bl status\grqq{}. & OK\\
	\hline
	Es wird angezeigt, dass Bluetooth am Handy eingeschaltet ist. & OK\\
	\hline
	Der Nutzer schaltet Bluetooth an seinem Smartphone aus. & OK\\
	\hline
	\hline
	Der Nutzer drückt den Button \glqq{}Get Bl status\grqq{}. & OK\\
	\hline
	Es wird angezeigt, dass Bluetooth am Handy ausgeschaltet ist. & OK\\
	\hline
\end{tabular}
\subsubsection{alles auf einmal}
alles auf einmal (soll ich das wirklich noch machen ??? es wird wahrscheinlich extrem lang )



\section{Regressionstests}

\section{Beschreibung Fehler}

\section{Anhang}
\subsection{Unit Test Reports}
\subsubsection{Bibliothek}
\subsubsection{App}

\subsection{Links}
\subsubsection{Bibliothek}
NuGet Package: \url{https://www.nuget.org/packages/EarablesBLE}\\
GitHub Repository: \url{https://github.com/vlle1/lib-earablesKIT}
\subsubsection{App}
GitHub Repository: \url{https://github.com/vlle1/earablesKIT}
%%%%%%%%%%%%%%%%%%%%%%%%%%%%%%%%%%%%%%% END CONTENT %%%%%%%%%%%%%%%%%%%%%%%%%%%%%%%%%%%%%%%%%%%


\printglossaries
\stepcounter{section}


\end{document}