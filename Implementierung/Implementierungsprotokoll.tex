\documentclass[a4paper,12pt]{article}
\usepackage{amssymb}
\usepackage{amsmath}
\usepackage[utf8]{inputenc} % Umlaute
\usepackage[ngerman]{babel} % Umlaute
\usepackage[T1]{fontenc}    % Umlaute
\usepackage[margin=2.5cm]{geometry}
\usepackage{booktabs}
\usepackage{lmodern}
\usepackage{titlesec}
\usepackage{longtable}
% Notwendig für Links im Text
\usepackage{hyperref}
%%\usepackage{svg}
% glossar, see http://en.wikibooks.org/wiki/LaTeX/Glossary
% muss NACH hyperref geladen werden, sonst funktionieren die Links nicht
\usepackage[toc]{glossaries}

% Kompatibilität
\ifx\pdftexversion\undefined
\usepackage[dvips]{graphicx}
\else
\usepackage[pdftex]{graphicx}
\DeclareGraphicsRule{*}{mps}{*}{}
\fi
\setlength{\parindent}{0pt}


%irgendwas mit section formatierung (titlesec package)
\titleformat{\paragraph}[hang]{\normalfont\normalsize\bfseries}{\theparagraph}{1em}{}
%%%%%%%%%%%%%%%%%%%%%%%%%%%%%%%%%%%%%%%%%%%%%%%%%%%%%%%%%%%%%%%%%%%%%%
% Variablen                                 						 %
%%%%%%%%%%%%%%%%%%%%%%%%%%%%%%%%%%%%%%%%%%%%%%%%%%%%%%%%%%%%%%%%%%%%%%
\newcommand{\authorName}{Tec O'Brain (Entwickler: David Höglinger, Jan Ettrich, Erwin Müller, Benedikt Rittner, Valentin Quapil)}
\newcommand{\auftraggeber}{Karlsruhe Institute of Technology (Teco)}
\newcommand{\auftragnehmer}{\authorName}
\newcommand{\projektName}{Implementierung Earables}
\newcommand{\tags}{\authorName, Architektur, Implementierung, KIT, Informatik, PSE}
\newcommand{\documentVersion}{1.0}
\title{\projektName}
\date{\today}
\author{Tec O'Brain}

%%%%%%%%%%%%%%%%%%%%%%%%%%%%%%%%%%%%%%%%%%%%%%%%%%%%%%%%%%%%%%%%%%%%%%
% PDF Meta information                                 				 %
%%%%%%%%%%%%%%%%%%%%%%%%%%%%%%%%%%%%%%%%%%%%%%%%%%%%%%%%%%%%%%%%%%%%%%
\hypersetup{
  pdfauthor   = {\authorName},
  pdfkeywords = {\tags},
  pdftitle    = {\projektName)}
}

%%%%%%%%%%%%%%%%%%%%%%%%%%%%%%%%%%%%%%%%%%%%%%%%%%%%%%%%%%%%%%%%%%%%%%
% Create a shorter version for tables. DO NOT CHANGE               	 %
%%%%%%%%%%%%%%%%%%%%%%%%%%%%%%%%%%%%%%%%%%%%%%%%%%%%%%%%%%%%%%%%%%%%%%
\newcommand\addrow[2]{#1 &#2\\ }

\newcommand\addheading[2]{#1 &#2\\ \hline}
\newcommand\tabularhead{\begin{tabular}{lp{13cm}}
\hline
}

\newcommand\addmulrow[2]{ \begin{minipage}[t][][t]{2.5cm}#1\end{minipage}%
   &\begin{minipage}[t][][t]{8cm}
    \begin{enumerate} #2   \end{enumerate}
    \end{minipage}\\ }

\newenvironment{usecase}{\tabularhead}
{\hline\end{tabular}}

\usepackage{microtype}
%%%%%%%%%%%%%%%%%%%%%%%%%%%%%%%%%%%%%%%%%%%%%%%%%%%%%%%%%%%%%%%%%%%%%%
% GLOSSARY ENTRIES                 	                              	 %
%%%%%%%%%%%%%%%%%%%%%%%%%%%%%%%%%%%%%%%%%%%%%%%%%%%%%%%%%%%%%%%%%%%%%%


%%%%%%%%%%%%%%%%%%%%%%%%%%%%%%%%%%%%%%%%%%%%%%%%%%%%%%%%%%%%%%%%%%%%%%
% THE DOCUMENT BEGINS             	                              	 %
%%%%%%%%%%%%%%%%%%%%%%%%%%%%%%%%%%%%%%%%%%%%%%%%%%%%%%%%%%%%%%%%%%%%%%
\begin{document}
\pagenumbering{roman}
 \begin{titlepage}
\maketitle
\thispagestyle{empty} % no page number

\begin{verbatim}












\end{verbatim}


  \begin{tabular}[t]{p{4 cm}p{8 cm}}
	Projekt:       & \projektName \\[1.2ex]
	Auftraggeber:  & \auftraggeber\\[1.2ex]
	Auftragnehmer: & \auftragnehmer\\[1.2ex]
  \end{tabular}


\begin{tabular}[t]{|p{4 cm}|p{8 cm}|}
\hline
\textbf{Datum} & \textbf{Autor(en)} \\
\hline
\hline
\today & \authorName \\
\hline
\end{tabular}
\end{titlepage}
         % Deckblatt.tex laden und einfügen
 \setcounter{page}{2}
 \tableofcontents          % Inhaltsverzeichnis ausgeben
 \clearpage
 \pagenumbering{arabic}
%%%%%%%%%%%%%%%%%%%%%%%%%%%%%%%%%%%%%%% CONTENT %%%%%%%%%%%%%%%%%%%%%%%%%%%%%%%%%%%%%%%%%%%%%%%

\section{Einleitung}
\section{Probleme und Änderungen am Entwurf}

\subsection{Bibliothek}

\subsubsection{Schnittstellen der Bibliothek}
\paragraph{Beschreibung}
Es war geplant, dass die meisten Schnittstellen der Bibliothek ein Boolean zurückgeben, der true ist, falls die Anweisung korrekt durchgeführt werden konnte und false , falls es Komplikationen gab.
Um besser auf Fehler und Exceptions zu reagieren macht es mehr Sinn, dass diese Methoden void sind und im Fehlerfall Exceptions schmeißen. Außerdem gibt es nun keine Get/Set Methoden mehr. Sie wurden durch Properties ersetzt.

\paragraph{Änderungen in Klasse IEarablesConnection}
\begin{itemize}
	\item[-] Folgende Methoden haben nicht mehr den Rückgabetypen bool sondern void und schmeißen im Fehlerfall Exceptions
	\item[] ConnectToDevice,
	\item[] DisconnectFromDevice,
	\item[] StartSampling,
	\item[] StopSampling,
	\item[] 

	\item[-] Properties anstatt Methoden für folgende Attribute
	\item[] Connected
	\item[] IsBluetoothActive
	\item[] SampleRate
	\item[] AccLPF
	\item[] GyroLPF
	\item[] BatteryVoltage
	\item[] 

	\paragraph{Änderungen in Klasse EarablesConnection}
	
	\item[-] Durch die Änderungen der Get/Set Methoden in Properties ergeben sich folgende neue private Methoden
	\item[] CheckIsBluetoothActive(): void
	\item[] GetAccelerometerLPF(): LPF\_Accelerometer
	\item[] SetAccelerometerLPF(LPF\_Accelerometer value): void
	\item[] GetGyroscopeLPF(): LPF\_Gyroscope
	\item[] SetGyroscopeLPF(LPF\_Gyroscope value): void
	\item[] 

	
	\item[-] Besonderheit des BatteryVoltage Attributes
	\item[] Die Methode GetBatteryVoltageFromDevice(): void wird automatisch aufgerufen wenn sich die BatteryVoltage der Earables ändert und überträgt den aktuellen Wert in das BatteryVoltage Attribut.
	\item[] Die Methode initBatteryVoltage(): Task liest den BatteryVoltage Wert der Earables zum ersten mal aus, für den Fall dass sie abgefragt wird bevor sie sich das erste mal aktualisiert hat.
\end{itemize}

\subsubsection{Exceptions ergänzt}
\paragraph{Beschreibung}
Aufgrund der Änderungen, dass die Methodenschnittstellen der Klasse IEarablesConnection nicht mehr Boolean zurückgeben sondern void und im Fehlerfall Exceptions schmeißen, ist es notwendig Exceptions zu ergänzen.
\paragraph{Folgende Exceptions ergänzt}
\begin{itemize}
	\item[-] AllreadyConnectedException. Sie wird geschmissen sobald versucht wird eine Verbindung mit einem Device herzustellen, obwohl schon eine Verbindung existiert.
	\item[-] ConnectionFailedException. Es kam vereinzelt auf, dass die App beim verbinden abgestürzt ist. Dies wird nun verhindert und stattdessen wird diese  Exception geschmissen.
	\item[-] InvalideSampleRateException. Die Samplerate muss zwischen 1 und 100 liegen, ansonsten wird die Exception geschmissen (Dies ist von den Earables vorgegeben).
	\item[-] NoConnectionException. Falls versucht wird eine Aktion durchzuführen, für die eine Verbindung zu einem Device benötigt wird, diese aber nicht existiert, wird diese Exception geschmissen.
\end{itemize}


\subsubsection{Attribute der Bibliothek}
\paragraph{Beschreibung}
Für die Attribute config und characteristics ist es, in Betracht der Datenkapselung, Sinnvoller sie private zu halten.
	\paragraph{Änderungen in Klasse IEarablesConnection}
\begin{itemize}
	\item[-] Die Attribute config und characters sind nun private anstatt public.
\end{itemize}


\subsubsection{Enums }
\paragraph{Beschreibung}
Die Enums können nicht mit Zahlen beginnen.
\paragraph{Änderungen in den Enums LPF\_Accelerometer und LPF\_Gyroscope}
\begin{itemize}
	 \item[-] Die Inhalte der Enums beginnen nun nicht mehr mit der Zahl sondern mit Hz. Beispielsweise nicht mehr 10Hz sondern Hz10.
\end{itemize}

\subsubsection{StartScanning Methoden der Klasse IEarablesConnection}
\paragraph{Beschreibung}
Es war ursprünglich gedacht, dass die StartScanning Methode eine Liste mit alle gefundenen Devices zurück gibt. Die Liste die zurückgegeben wurde war aber immer leer, da Methoden aus dem Plugin.BLE benutzt werde, die asyncron laufen. So wurde zuerst die Liste zurückgegeben und danach wurde sie befüllt. Nach etlichen gescheiterten Versuchen auf die Threads und Tasks zu warten haben wir uns dafür entschieden jedes mal ein Event zu schmeißen, wenn ein neues Device gescannt wurde. So entstand das neue Event NewDeiviceFound. Es wird geschmissen sobald ein neues Device, beim Scannen gefunden wurde. Die Argumente des Events beinhalten das neue Device.
\paragraph{Änderungen in der Klasse IEarablesConnection}
\begin{itemize}
	 \item[-] public event EventHandler<NewDeviceFoundArgs> NewDeviceFound hinzugefügt
\end{itemize}
\paragraph{Klasse NewDeviceFound hinzugefügt}
\begin{itemize}
	 \item[-] Enthält als einziges Attribut Device vom Typen IDevice und ein Konstruktor zum setzen des Attributs.
\end{itemize}

\subsubsection{Static Methoden in der Klasse IEarablesConnection}
\paragraph{Beschreibung}
Es wurde angenommen, dass die Methoden die auf die Events der Earables reagieren static sein müssen. Diese Annahme war falsch.
\paragraph{Änderungen in der Klasse IEarablesConnection}
\begin{itemize}
	 \item[-] static Methoden sind nun nicht mehr static
\end{itemize}

\subsubsection{Klasse EarablesConnection reagiert auf Verbinden und trennen von Devices}
\paragraph{Beschreibung}
Es wurden zwei Methoden hinzugefügt, um den Connectionstate des Smartphones, mit Hilfe des DeviceConnectionStateChanged Events, weiterzuleiten. Sie werden aktiviert wenn sich die Earables mutwillig vom Device trennen oder die Verbindung unterbrochen wird.
\paragraph{Änderungen in der Klasse EarablesConnection}
\begin{itemize}
	 \item[-] OnDeviceDisconnected(object sender, Plugin.BLE.Abstractions.EventArgs.DeviceEventArgs args): void ergänzt
	\item[-]  OnDeviceConnectionLost(object sender, DeviceErrorEventArgs args): void ergänzt
\end{itemize}

\subsubsection{IMUDataExtructor}
\paragraph{Beschreibung}
Die Methode ExtractIMUDataString benötigt für die Berechnung zusätzlich noch den Offset. Dieser wird als vierter Parameter in Form eines Bytearrays mitgegeben. Außerdem sind die ScaleFactoren, die als Parameter übergeben werden jetzt nicht mehr vom Typen int sondern double, da es sich hierbei um Fließkommazahlen handeln kann. Dementsprechend hat sich der Returntyp der Methode ExctractIMUScaleFactorGyroscope in double geändert.
Die Klasse wurde um zwei Methoden ergänzt, die die Range des Accelerometers und des Gyroscopes aus dem entsprechenden Bytearray ermitteln.
\paragraph{Änderungen in der Klasse IMUDataExctructor}
\begin{itemize}
	 \item[-]  ExtractIMUDataString besitzt nun einen vierten Parameter vom Typ byte[]. Der zweite und dritte Parameter ist nun vom Typen double anstatt int.
	\item[-] ExctractIMUScaleFactorGyroscope hat jetzt als Returntyp double und nicht mehr int.
	\item[-] ExtractIMURangeAccelerometer(byte[] bytes): int ergänzt. Diese Methode ermittelt die Range für den Accelerometer.
	\item[-] ExtractIMURangeGyroscope(byte[] bytes): double ergänzt. Diese Methode ermittelt die Range für das Gyroscope.
\end{itemize}

\subsubsection{Charakteristik umbenannt}
\paragraph{Beschreibung}
Eine Charakteristik wurde umbenannt.
\paragraph{Änderungen in der Klasse Constants}
\begin{itemize}
	\item[-] Attribut IMU\_FULL\_SCALE\_RANG\_CHAR in OFFSET\_CHAR umbenannt.
\end{itemize}
\paragraph{Änderungen in der Klasse Characteristics}
\begin{itemize}
	\item[-] Attribute IMUScaleRangeChar in OffsetChar umbenannt.
\end{itemize}

\subsubsection{Connection überprüfung}
\paragraph{Beschreibung}
Um festzustellen ob eine Verbindung zu einem Device besteht wurde eine Methode hinzugefügt
\paragraph{Änderungen in der Klasse EarablesConnection}
\begin{itemize}
	\item[-] Methode CheckConnection():void hinzugefügt. Sie wird am Anfang aller  Methode, die eine Verbindung voraussetzen, aufgerufen und prüft ob eine Verbindung existiert. Falls keine existiert schmeißt sie eine die Exception NoConnectionException.
\end{itemize}

\subsubsection{Klasse ConfigContainer ergänzt}
\paragraph{Beschreibung}
Um das Bytearray des IMU auszuwerten benötigt man die Scalefactoren für den Accelerometer und für das Gyroscope. Wenn man diesen aber erst in der Methode OnValueUpdatedIMU ausliest stürzt die App ab, da die Methode jedes mal aufgerufen wird, wenn neue IMUDaten anliegen, was bis zu 100 mal pro Sekunde passieren kann. Das liegt daran, dass das Auslesen der Charakteristiken mit Hilfe von asynchronen Methoden aus dem Plugin.BLE geschieht. Deswegen werden die Scalefactoren bereits beim Starten des Samplings ausgelesen und müssen irgendwo zwischengespeichert werden. Dafür wurden in der Klasse ConfigContainer zwei Attribute AccScaleFactor und GyroScaleFactor hinzugefügt. Außerdem enthält sie zusätzlich noch eine Methode um zu überprüfen, ob die Samplingrate im gültigen Wertebereich liegt.
\paragraph{Änderungen in der Klasse ConfigContainer}
\begin{itemize}
	\item[-] AccScaleFactor: int und GyroScaleFactor: int hinzugefügt.
	\item[-] private setSamplingRate(int rate): void hinzugefügt. Beim setzten des Samplingrate Propertys wird diese Methode ausgeführt und stellt sicher, dass nur gültige Werte angenommen werden.
\end{itemize}

\subsection{Erweiterungsmodul}

\subsection{Model (andere Bestandteile)}

\subsection{Views und ViewModel}
\subsubsection{Commandstruktur bei Modi}
\paragraph{Beschreibung}
Da beim Aktivieren des Modus' die Seite gewechselt werden muss und das ViewModel die Views nicht kennen darf, muss der Button Click in den Code Behind delegiert werden und danach erst ins Viewmodel.\\
Somit wurden die Commands ersetzt durch blablabla.
\paragraph{Änderungen in Klasse BaseModeViewModel}
\begin{itemize}
  \item[-] methodeXY() hinzugefügt
  \item[] CommandXY entwfernt (vernünftige Signaturen verwenden!)
\end{itemize}
\paragraph{Änderungen in Klassen CountViewModel, StepViewModel}
Die Kindklassen ändern sich entsprechend folgendermaßen:
\begin{itemize}
  \item[-] methodeXY() hinzugefügt
  \item[] CommandXY entwfernt (vernünftige Signaturen verwenden!)
\end{itemize}
\section{Statistik der Testfälle}

\section{Implementierungsplan}

\section{Anhang}


%%%%%%%%%%%%%%%%%%%%%%%%%%%%%%%%%%%%%%% END CONTENT %%%%%%%%%%%%%%%%%%%%%%%%%%%%%%%%%%%%%%%%%%%


\printglossaries
\stepcounter{section}


\end{document}
