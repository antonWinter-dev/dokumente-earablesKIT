\newglossaryentry{Echtzeit}{name=Echtzeit, description={Bereitstellen/Anzeigen von Daten mit einer durch die Verarbeitung bedingten Verzögerung von bis zu ca. 2 Sekunden zwischen dem Anfallen der (Roh-)Daten und der Ausgabe bzw. Visualisierung.}}
\newglossaryentry{Vorgang}{name=Vorgang, description={Als Vorgang wird in diesem Pflichtenheft bezeichnet, wenn ein Modus ausgewählt ist und Start gedrückt wurde. Der Vorgang endet mit dem Drücken von Stopp bzw. wird mit dem Wechseln des Modus.}}
\newglossaryentry{BLE}{name=BLE, description={Bluetooth Low Energy ist eine Technologie, die Teil des Industriestandards Bluetooth ist und eine energiesparende, kabellose Kommunikation zwischen Geräten in einer Entfernung von bis zu ca. 10 Metern ermöglicht.}}
\newglossaryentry{Wearable Computer}{name=Wearable Computer, description={Unter dem Begriff Wearable Computer versteht man Computersysteme, die am Körper, unter der Kleidung oder als Implantat unter der Haut getragen werden können.}}
\newglossaryentry{IMU}{name=6-Achsen IMU, description={Ein 6-Achsen IMU ist ein Beschleunigungssensor mit Gyroskop.}}
\newglossaryentry{GUI}{name=GUI, description={GUI ist die Abkürzung für den englischen Begriff \glqq graphical user interface\grqq . Sie ist die Schnittstelle zwischen Mensch und Maschine und ermöglicht dem Nutzer die Eingabe/Steuerung, der Maschine.}}
\newglossaryentry{CPB}{name=Cross-Platform Bibliothek, description={Eine Cross-Platform Bibliothek ist nichts weiter als eine Bibliothek die auf Rechnersystemen mit verschiedener Architektur laufen kann.}}
\newglossaryentry{Steuerungsparameter}{name=Steuerungsparameter, description={Unter Steuerungsparametern fassen wir die Länge der Verbindungsintervalle zwischen Earables und Smartphone sowie die Abtastrate und den Wertebereich von integriertem Gyroskop und Beschleunigungssensor zusammen.}}
\newglossaryentry{Schrittfrequenz}{name=Schrittfrequenz, description={Die Schrittfrequenz gibt an wie viele Schritte pro Zeiteinheit gemacht werden.}}
\newglossaryentry{Rohdaten}{name=Rohdaten, description={Als Rohdaten werden unverarbeitete Daten bezeichnet.}}
\newglossaryentry{TTS}{name=Text-To-Speech, description={Ein Text-to-Speech-System (TTS) (oder Vorleseautomat) wandelt Fließtext in eine akustische Sprachausgabe um. Dabei erfolgt diese auf Deutsch oder auf Englisch, abhängig davon, was als Sprache eingestellt ist.}}
\newglossaryentry{Earables}{name=Earables, description={Eine Zusammenschließung des Wortes Wearable und Earphone. Dabei handelt es sich um Kopfhörer, die mit Sensoren ausgestattet sind.}}
\newglossaryentry{Vorgangsdaten}{name=Vorgangsdaten, description={Daten, die bei der Ausführung eines Vorgangs gespeichert werden (z.B. Schritte, Sit-ups,\dots).}}
\newglossaryentry{Datenbank}{name={Datenbank},
	description={Die Datenbank speichert die Trainingsdaten in Form von DBEntries. Dabei wird die Datenbank von dem Plugin SQLite benutzt.}}
\newglossaryentry{CSV}{name={CSV},
	description={CSV steht für "Comma-Seperated-Value" ein Dateityp, bei dem die Attribute durch ein Komma getrennt werden. Eine Zeile bildet dabei immer einen Eintrag ab. Das Format ist human-readable und veränderbar.}}
\newglossaryentry{Rg.Plugins.Popup}{name={Rg.Plugins.Popup},
	description={Rg.Plugins.Popup ist eine Erweiterung (Plugin), welche es einem ermöglicht Pop-up Fenster zu erstellen und anzuzeigen. Dabei ist das Plugin über den Pluginmanager 'NuGet' einbindbar.}}
\newglossaryentry{DependencyInjection}{name={Microsoft.Extensions.DependencyInjection},
	description={Microsoft.Extensions.DependencyInjection ist eine Erweiterung, welches ermöglicht Komponenten als Services zu registrieren und sie somit abgekapselt und modularer zu betrachten. Mit Klassen, wie der ServiceCollection, oder dem ServiceProvicer, bietet die Erweiterung Microsoft.Extensions.DependencyInjection das Framework.}}
