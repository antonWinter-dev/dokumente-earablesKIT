\documentclass[a4paper,12pt]{article}
\usepackage{amssymb}
\usepackage{amsmath}
\usepackage[utf8]{inputenc} % Umlaute
\usepackage[ngerman]{babel} % Umlaute
\usepackage[T1]{fontenc}    % Umlaute
\usepackage[margin=2.5cm]{geometry}
\usepackage{booktabs}
\usepackage{lmodern}
\usepackage{titlesec}
% Notwendig für Links im Text
\usepackage{hyperref}

% glossar, see http://en.wikibooks.org/wiki/LaTeX/Glossary
% muss NACH hyperref geladen werden, sonst funktionieren die Links nicht
\usepackage[toc]{glossaries}

% Kompatibilität
\ifx\pdftexversion\undefined
\usepackage[dvips]{graphicx}
\else
\usepackage[pdftex]{graphicx}
\DeclareGraphicsRule{*}{mps}{*}{}
\fi

% specify the path to the images
\graphicspath{{bilder/}}

%irgendwas mit section formatierung (titlesec package)
\titleformat{\paragraph}[hang]{\normalfont\normalsize\bfseries}{\theparagraph}{1em}{}
%%%%%%%%%%%%%%%%%%%%%%%%%%%%%%%%%%%%%%%%%%%%%%%%%%%%%%%%%%%%%%%%%%%%%%
% Variablen                                 						 %
%%%%%%%%%%%%%%%%%%%%%%%%%%%%%%%%%%%%%%%%%%%%%%%%%%%%%%%%%%%%%%%%%%%%%%
\newcommand{\authorName}{Tec O'Brain (Entwickler: David Höglinger, Jan Ettrich, Erwin Müller, Benedikt Rittner, Valentin Quapil)}
\newcommand{\auftraggeber}{Karlsruhe Institute of Technology (Teco)}
\newcommand{\auftragnehmer}{\authorName}
\newcommand{\projektName}{Entwurf Earables}
\newcommand{\tags}{\authorName, Architektur, Entwurf, KIT, Informatik, PSE}
\newcommand{\glossarName}{Glossar}
\newcommand{\documentVersion}{0.1}
\title{\projektName}
\date{\today}
\author{Tec O'Brain}

%%%%%%%%%%%%%%%%%%%%%%%%%%%%%%%%%%%%%%%%%%%%%%%%%%%%%%%%%%%%%%%%%%%%%%
% PDF Meta information                                 				 %
%%%%%%%%%%%%%%%%%%%%%%%%%%%%%%%%%%%%%%%%%%%%%%%%%%%%%%%%%%%%%%%%%%%%%%
\hypersetup{
  pdfauthor   = {\authorName},
  pdfkeywords = {\tags},
  pdftitle    = {\projektName)}
}

%%%%%%%%%%%%%%%%%%%%%%%%%%%%%%%%%%%%%%%%%%%%%%%%%%%%%%%%%%%%%%%%%%%%%%
% Create a shorter version for tables. DO NOT CHANGE               	 %
%%%%%%%%%%%%%%%%%%%%%%%%%%%%%%%%%%%%%%%%%%%%%%%%%%%%%%%%%%%%%%%%%%%%%%
\newcommand\addrow[2]{#1 &#2\\ }

\newcommand\addheading[2]{#1 &#2\\ \hline}
\newcommand\tabularhead{\begin{tabular}{lp{13cm}}
\hline
}

\newcommand\addmulrow[2]{ \begin{minipage}[t][][t]{2.5cm}#1\end{minipage}%
   &\begin{minipage}[t][][t]{8cm}
    \begin{enumerate} #2   \end{enumerate}
    \end{minipage}\\ }

\newenvironment{usecase}{\tabularhead}
{\hline\end{tabular}}

\usepackage{microtype}
%%%%%%%%%%%%%%%%%%%%%%%%%%%%%%%%%%%%%%%%%%%%%%%%%%%%%%%%%%%%%%%%%%%%%%
% GLOSSARY ENTRIES                 	                              	 %
%%%%%%%%%%%%%%%%%%%%%%%%%%%%%%%%%%%%%%%%%%%%%%%%%%%%%%%%%%%%%%%%%%%%%%

\makeglossaries
\loadglsentries{Glossar.tex}

%%%%%%%%%%%%%%%%%%%%%%%%%%%%%%%%%%%%%%%%%%%%%%%%%%%%%%%%%%%%%%%%%%%%%%
% THE DOCUMENT BEGINS             	                              	 %
%%%%%%%%%%%%%%%%%%%%%%%%%%%%%%%%%%%%%%%%%%%%%%%%%%%%%%%%%%%%%%%%%%%%%%
\begin{document}
\pagenumbering{roman}
 \begin{titlepage}
\maketitle
\thispagestyle{empty} % no page number

\begin{verbatim}












\end{verbatim}


  \begin{tabular}[t]{p{4 cm}p{8 cm}}
	Projekt:       & \projektName \\[1.2ex]
	Auftraggeber:  & \auftraggeber\\[1.2ex]
	Auftragnehmer: & \auftragnehmer\\[1.2ex]
  \end{tabular}


\begin{tabular}[t]{|p{4 cm}|p{8 cm}|}
\hline
\textbf{Datum} & \textbf{Autor(en)} \\
\hline
\hline
\today & \authorName \\
\hline
\end{tabular}
\end{titlepage}
         % Deckblatt.tex laden und einfügen
 \setcounter{page}{2}
 \tableofcontents          % Inhaltsverzeichnis ausgeben
 \clearpage
 \pagenumbering{arabic}
%%%%%%%%%%%%%%%%%%%%%%%%%%%%%%%%%%%%%%% CONTENT %%%%%%%%%%%%%%%%%%%%%%%%%%%%%%%%%%%%%%%%%%%%%%%

\section{Einleitung}
In diesem Dokument wird der Entwurf der \Gls{CPB}, des Erweiterungsmoduls und der App spezifiziert. Außerdem wird die Interaktion der einzelnen Komponenten beschrieben.
Die verwendete Notation richtet sich nach dem UML-Standard.

\section{Aufbau}
    \subsection{Architektur}
    \subsection{Klassendiagramm}


\section{Klassenübersicht}
\section{Klassenbeschreibung Model}
\subsection{Settings Service}
\subsubsection{User}
    \paragraph{Klassenbeschreibung:}
    Die User-Klasse spezifiziert den Benutzer, der die App gerade verwendet.\\
    \paragraph{Attribute:}
    \begin{tabular}{p{5cm}p{12cm}}
        + String Name & Der Name des Benutzers.\\
        + int StepLength & Die durchschnittliche Schrittlänge des Benutzers in cm.\\
    \end{tabular}
    \paragraph{Methoden:}
    -
\section{Klassenbeschreibung View-Model}
\section{Klassenbeschreibung View}
\section{Interaktionsdiagramme}
\subsection{Aktivitätsdiagramm Lauschen und Agieren}
\subsection{Sequenzdiagramme}
%später:
\subsubsection{Abläufe in der App}
Programmstart
Sprache Ändern
\section{Entwurfdaten}
\subsection{Ressourceenverzeichnis}
\subsection{lokale Datenbank}
\subsection{App Properties}

\section{Klassenindex}
%macht valle ganz am Ende oder wir lassen es weg
\section{Anhang}

%%%%%%%%%%%%%%%%%%%%%%%%%%%%%%%%%%%%%%% END CONTENT %%%%%%%%%%%%%%%%%%%%%%%%%%%%%%%%%%%%%%%%%%%
\clearpage
\printglossaries
\stepcounter{section}


\end{document}