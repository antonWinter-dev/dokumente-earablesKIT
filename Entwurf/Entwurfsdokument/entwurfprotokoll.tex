\documentclass[a4paper,12pt]{article}
\usepackage{amssymb}
\usepackage{amsmath}
\usepackage[utf8]{inputenc} % Umlaute
\usepackage[ngerman]{babel} % Umlaute
\usepackage[T1]{fontenc}    % Umlaute
\usepackage[margin=2.5cm]{geometry}
\usepackage{booktabs}
\usepackage{lmodern}
\usepackage{titlesec}
% Notwendig für Links im Text
\usepackage{hyperref}

% glossar, see http://en.wikibooks.org/wiki/LaTeX/Glossary
% muss NACH hyperref geladen werden, sonst funktionieren die Links nicht
\usepackage[toc]{glossaries}

% Kompatibilität
\ifx\pdftexversion\undefined
\usepackage[dvips]{graphicx}
\else
\usepackage[pdftex]{graphicx}
\DeclareGraphicsRule{*}{mps}{*}{}
\fi

% specify the path to the images
\graphicspath{{bilder/}}

%irgendwas mit section formatierung (titlesec package)
\titleformat{\paragraph}[hang]{\normalfont\normalsize\bfseries}{\theparagraph}{1em}{}
%%%%%%%%%%%%%%%%%%%%%%%%%%%%%%%%%%%%%%%%%%%%%%%%%%%%%%%%%%%%%%%%%%%%%%
% Variablen                                 						 %
%%%%%%%%%%%%%%%%%%%%%%%%%%%%%%%%%%%%%%%%%%%%%%%%%%%%%%%%%%%%%%%%%%%%%%
\newcommand{\authorName}{Tec O'Brain (Entwickler: David Höglinger, Jan Ettrich, Erwin Müller, Benedikt Rittner, Valentin Quapil)}
\newcommand{\auftraggeber}{Karlsruhe Institute of Technology (Teco)}
\newcommand{\auftragnehmer}{\authorName}
\newcommand{\projektName}{Entwurf Earables}
\newcommand{\tags}{\authorName, Architektur, Entwurf, KIT, Informatik, PSE}
\newcommand{\glossarName}{Glossar}
\newcommand{\documentVersion}{0.1}
\title{\projektName}
\date{\today}
\author{Tec O'Brain}

%%%%%%%%%%%%%%%%%%%%%%%%%%%%%%%%%%%%%%%%%%%%%%%%%%%%%%%%%%%%%%%%%%%%%%
% PDF Meta information                                 				 %
%%%%%%%%%%%%%%%%%%%%%%%%%%%%%%%%%%%%%%%%%%%%%%%%%%%%%%%%%%%%%%%%%%%%%%
\hypersetup{
  pdfauthor   = {\authorName},
  pdfkeywords = {\tags},
  pdftitle    = {\projektName)}
}

%%%%%%%%%%%%%%%%%%%%%%%%%%%%%%%%%%%%%%%%%%%%%%%%%%%%%%%%%%%%%%%%%%%%%%
% Create a shorter version for tables. DO NOT CHANGE               	 %
%%%%%%%%%%%%%%%%%%%%%%%%%%%%%%%%%%%%%%%%%%%%%%%%%%%%%%%%%%%%%%%%%%%%%%
\newcommand\addrow[2]{#1 &#2\\ }

\newcommand\addheading[2]{#1 &#2\\ \hline}
\newcommand\tabularhead{\begin{tabular}{lp{13cm}}
\hline
}

\newcommand\addmulrow[2]{ \begin{minipage}[t][][t]{2.5cm}#1\end{minipage}%
   &\begin{minipage}[t][][t]{8cm}
    \begin{enumerate} #2   \end{enumerate}
    \end{minipage}\\ }

\newenvironment{usecase}{\tabularhead}
{\hline\end{tabular}}

\usepackage{microtype}
%%%%%%%%%%%%%%%%%%%%%%%%%%%%%%%%%%%%%%%%%%%%%%%%%%%%%%%%%%%%%%%%%%%%%%
% GLOSSARY ENTRIES                 	                              	 %
%%%%%%%%%%%%%%%%%%%%%%%%%%%%%%%%%%%%%%%%%%%%%%%%%%%%%%%%%%%%%%%%%%%%%%

\makeglossaries
\loadglsentries{Glossar.tex}

%%%%%%%%%%%%%%%%%%%%%%%%%%%%%%%%%%%%%%%%%%%%%%%%%%%%%%%%%%%%%%%%%%%%%%
% THE DOCUMENT BEGINS             	                              	 %
%%%%%%%%%%%%%%%%%%%%%%%%%%%%%%%%%%%%%%%%%%%%%%%%%%%%%%%%%%%%%%%%%%%%%%
\begin{document}
\pagenumbering{roman}
 \begin{titlepage}
\maketitle
\thispagestyle{empty} % no page number

\begin{verbatim}












\end{verbatim}


  \begin{tabular}[t]{p{4 cm}p{8 cm}}
	Projekt:       & \projektName \\[1.2ex]
	Auftraggeber:  & \auftraggeber\\[1.2ex]
	Auftragnehmer: & \auftragnehmer\\[1.2ex]
  \end{tabular}


\begin{tabular}[t]{|p{4 cm}|p{8 cm}|}
\hline
\textbf{Datum} & \textbf{Autor(en)} \\
\hline
\hline
\today & \authorName \\
\hline
\end{tabular}
\end{titlepage}
         % Deckblatt.tex laden und einfügen
 \setcounter{page}{2}
 \tableofcontents          % Inhaltsverzeichnis ausgeben
 \clearpage
 \pagenumbering{arabic}
%%%%%%%%%%%%%%%%%%%%%%%%%%%%%%%%%%%%%%% CONTENT %%%%%%%%%%%%%%%%%%%%%%%%%%%%%%%%%%%%%%%%%%%%%%%

\section{Einleitung}
In diesem Dokument wird der Entwurf der \Gls{CPB}, des Erweiterungsmoduls und der App spezifiziert. Außerdem wird die Interaktion der einzelnen Komponenten beschrieben.
Die verwendete Notation richtet sich nach dem UML-Standard.

\section{Aufbau}
    \subsection{Architektur}
    \subsection{Klassendiagramm}


\section{Klassenübersicht}
\section{Klassenbeschreibung Model}
\subsection{ServiceManager}
	\paragraph{Klassenbeschreibung:}
	Der ServiceManager verwaltet alle Services des Models mithilfe eines Serviceproviders. Dies wird von anderen Services des Models, sowie von dem Viewmodel benutzt. Er dient zur Delegierung der Services.\\ 
	Dabei implementiert der ServiceManager das Interface IManager.
	Von diesem bekommt er die Methode serviceRegistration() übergeben.
	Der ServiceManager speichert die Services als Referenzen und achtet, dass diese ein Singleton sind. (Es existiert immer nur eine Instanz des Services).
	Er selbst implementriert das Singleton Muster.
	
	\paragraph{Attribute:}
	\begin{tabular}{p{7cm}p{10cm}}
		- static instance : ServiceManager & Die Singleton Instanz, welche benutzt wird \\
		+ static Instance : ServiceManager & Regelt die Initialisierung von instance und gibt diese zurück\\
		+ serviceProvider : ServiceProvider & Enthält alle Referenzen auf die Services. Per GetService<T> wird der Service vom Typ T zurückgeliefert.\\
	\end{tabular}
	\paragraph{Methoden:}
	\begin{tabular}{p{7cm}p{10cm}}
		- Servicemanager() : void & privater Konstruktor, genutzt im Singleton Muster. Aufrufen der Methode serviceRegistration() zur Erstmaligen Registrierung der Services mit einer IServiceCollection und zur Erstellung des ServiceProvider.\\
	\end{tabular}
		
\subsection{Settings Service}
\subsubsection{User}
    \paragraph{Klassenbeschreibung:}
    Die User-Klasse spezifiziert den Benutzer, der die App gerade verwendet.\\
    \paragraph{Attribute:}
    \begin{tabular}{p{5cm}p{12cm}}
        + username: String & Der Name des Benutzers.\\
        + steplength: String & Die durchschnittliche Schrittlänge des Benutzers in cm.\\
    \end{tabular}
    \paragraph{Methoden:}
    \begin{tabular}{p{7cm}p{10cm}}
        + toString(): String & Wandelt das Objekt in einen String um. (vergleichbar mit JSON).\\
        + static parseUser(user: String): User & Wandelt JSON-String wieder in Objekt um.\\
    \end{tabular}
\subsection{ISettingsService}
	\paragraph{Interfacebeschreibung:}
	Das Interface ISettingsService bietet eine Schnittstelle für die Einstellungen der App. Sie implementiert Konstanten zur Identifizierung der Einstellungen und hält die aktuellen Einstellungen als Attribute.
	\paragraph{Attribute:}
	\begin{tabular}{p{7cm}p{10cm}}
		+ activeLanguage: CultureInfo & Die aktuelle Sprache der App (Deutsch oder Englisch)\\
		+ samplingRate: SamplingRate & Die aktuelle Samplingrate der \Gls{Earables} \\ 
		- LANGUAGE\_PROPERTY: String & Konstanter Bezeichner für die Einstellung der Sprache \\
		- USER\_PROPERTY: String & Konstanter Bezeichner für die Einstellung des Nutzers \\
		- SAMPLINGRATE\_PROPERTY: String & Konstanter Bezeichner für die Einstellung der Samplingrate \\
	\end{tabular}
	\paragraph{Methoden:}
	\begin{tabular}{p{7cm}p{10cm}}
		- loadSettings():void & Lädt alle Settings aus den Einstellungen in die Attribute.	
	\end{tabular}
\subsection{SettingsService}
	\paragraph{Klassenbeschreibung}
	Die Klasse SettingsService implementiert die Schnittstelle ISettingsService. In der Implementierung der Funktionen, welche die Speicherung der Einstellungen enthalten, wird die Klasse App.Current.Properties verwendet.
	Dabei wird immer sofort nach einer Änderung die neue Einstellung gespeichert und nicht erst beim Beenden der Sitzung.
	
	
	TODO: Sollen auch bei den konkreten Klassen die Methoden/Attribute nochmal aufgelistet werden?
\subsection{IDataBaseConnection}
	

\section{Klassenbeschreibung View-Model}
\section{Klassenbeschreibung View}
\section{Interaktionsdiagramme}
\subsection{Aktivitätsdiagramm Lauschen und Agieren}
\subsection{Sequenzdiagramme}
%später:
\subsubsection{Abläufe in der App}
Programmstart
Sprache Ändern
\section{Entwurfdaten}
\subsection{Ressourceenverzeichnis}
\subsection{lokale Datenbank}
\subsection{App Properties}

\section{Klassenindex}
%macht valle ganz am Ende oder wir lassen es weg
\section{Anhang}

%%%%%%%%%%%%%%%%%%%%%%%%%%%%%%%%%%%%%%% END CONTENT %%%%%%%%%%%%%%%%%%%%%%%%%%%%%%%%%%%%%%%%%%%
\clearpage
\printglossaries
\stepcounter{section}


\end{document}