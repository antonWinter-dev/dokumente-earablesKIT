\documentclass[a4paper,12pt]{article}
\usepackage{amssymb} % needed for math
\usepackage{amsmath} % needed for math
\usepackage[utf8]{inputenc} % this is needed for german umlauts
\usepackage[ngerman]{babel} % this is needed for german umlauts
\usepackage[T1]{fontenc}    % this is needed for correct output of umlauts in pdf
\usepackage[margin=2.5cm]{geometry} %layout
\usepackage{booktabs}

% this is needed for forms and links within the text
\usepackage{hyperref}

% glossar, see http://en.wikibooks.org/wiki/LaTeX/Glossary
% has to be loaded AFTER hyperref so that entries are clickable
\usepackage[nonumberlist]{glossaries}

% The following is needed in order to make the code compatible
% with both latex/dvips and pdflatex.
\ifx\pdftexversion\undefined
\usepackage[dvips]{graphicx}
\else
\usepackage[pdftex]{graphicx}
\DeclareGraphicsRule{*}{mps}{*}{}
\fi

\makeglossary

%%%%%%%%%%%%%%%%%%%%%%%%%%%%%%%%%%%%%%%%%%%%%%%%%%%%%%%%%%%%%%%%%%%%%%
% Variablen                                 						 %
%%%%%%%%%%%%%%%%%%%%%%%%%%%%%%%%%%%%%%%%%%%%%%%%%%%%%%%%%%%%%%%%%%%%%%
\newcommand{\authorName}{Worschiboiz}
\newcommand{\auftraggeber}{Karlsruhe Institute of Technology (IPD Schnellting)}
\newcommand{\auftragnehmer}{Teco}
\newcommand{\projektName}{Earables}
\newcommand{\tags}{\authorName, Pflichtenheft, KIT, Informatik, PSE}
\newcommand{\glossarName}{Glossar}
\title{\projektName~(Pflichtenheft)}
\author{\authorName}
\date{\today}

%%%%%%%%%%%%%%%%%%%%%%%%%%%%%%%%%%%%%%%%%%%%%%%%%%%%%%%%%%%%%%%%%%%%%%
% PDF Meta information                                 				 %
%%%%%%%%%%%%%%%%%%%%%%%%%%%%%%%%%%%%%%%%%%%%%%%%%%%%%%%%%%%%%%%%%%%%%%
\hypersetup{
  pdfauthor   = {\authorName},
  pdfkeywords = {\tags},
  pdftitle    = {\projektName~(Pflichtenheft)}
}

%%%%%%%%%%%%%%%%%%%%%%%%%%%%%%%%%%%%%%%%%%%%%%%%%%%%%%%%%%%%%%%%%%%%%%
% Create a shorter version for tables. DO NOT CHANGE               	 %
%%%%%%%%%%%%%%%%%%%%%%%%%%%%%%%%%%%%%%%%%%%%%%%%%%%%%%%%%%%%%%%%%%%%%%
\newcommand\addrow[2]{#1 &#2\\ }

\newcommand\addheading[2]{#1 &#2\\ \hline}
\newcommand\tabularhead{\begin{tabular}{lp{13cm}}
\hline
}

\newcommand\addmulrow[2]{ \begin{minipage}[t][][t]{2.5cm}#1\end{minipage}%
   &\begin{minipage}[t][][t]{8cm}
    \begin{enumerate} #2   \end{enumerate}
    \end{minipage}\\ }

\newenvironment{usecase}{\tabularhead}
{\hline\end{tabular}}

\usepackage{microtype}


%%%%%%%%%%%%%%%%%%%%%%%%%%%%%%%%%%%%%%%%%%%%%%%%%%%%%%%%%%%%%%%%%%%%%%
% THE DOCUMENT BEGINS             	                              	 %
%%%%%%%%%%%%%%%%%%%%%%%%%%%%%%%%%%%%%%%%%%%%%%%%%%%%%%%%%%%%%%%%%%%%%%
\begin{document}
 \pagenumbering{roman}
 \begin{titlepage}
\maketitle
\thispagestyle{empty} % no page number

\begin{verbatim}












\end{verbatim}


  \begin{tabular}[t]{p{4 cm}p{8 cm}}
	Projekt:       & \projektName \\[1.2ex]
	Auftraggeber:  & \auftraggeber\\[1.2ex]
	Auftragnehmer: & \auftragnehmer\\[1.2ex]
  \end{tabular}


\begin{tabular}[t]{|p{4 cm}|p{8 cm}|}
\hline
\textbf{Datum} & \textbf{Autor(en)} \\
\hline
\hline
\today & \authorName \\
\hline
\end{tabular}
\end{titlepage}
         % Deckblatt.tex laden und einfügen
 \setcounter{page}{2}
 \tableofcontents          % Inhaltsverzeichnis ausgeben
 \clearpage
 \pagenumbering{arabic}

\section{Einleitung}

\section{Zielbestimmung}
\subsection{Musskriterien}
\subsection{Wunschkriterien}
\subsection{Abgrenzungskriterien}
% Was will ich bewusst nicht umsetzen?
% Was soll es nicht sein?

\section{Produkteinsatz}
% Zielgruppe
% Anwendungsbereiche
% Betriebsbedinugen
% Wer? Was? Wozu?

\section{Produktumgebung}
% Unter welcher Software / Hardware läuft es?

\section{Funktionale Anforderungen}
% Was soll das Produkt machen können

\section{Produktdaten}
% Was soll gespeichert werden?

\subsection{Personendaten}
\subsection{Messdaten}

\section{Nichtfunktionale Anforderungen}
\subsection{Laufzeitverhalten}
\subsection{Speicherplatz}
\subsection{Stabilität}

\section{Systemmodelle}

\section{Benutzeroberfläche}

\section{Globale Testfälle und Szenarien}
\subsection{Globale Testfälle}
\subsubsection{Grundlegende Testfälle}
\subsubsection{Erweiterte Testfälle}
\subsection{Szenarien}
\section{Entwicklungsumgebung}

\clearpage
\newglossaryentry{Echtzeit}{name=Echtzeit, description={Bereitstellen/Anzeigen von Daten mit einer durch die Verarbeitung bedingten Verzögerung von bis zu ca. 2 Sekunden zwischen dem Anfallen der (Roh-)Daten und der Ausgabe bzw. Visualisierung.}}
\newglossaryentry{Vorgang}{name=Vorgang, description={Als Vorgang wird in diesem Pflichtenheft bezeichnet, wenn ein Modus ausgewählt ist und Start gedrückt wurde. Der Vorgang endet mit dem Drücken von Stopp bzw. wird mit dem Wechseln des Modus.}}
\newglossaryentry{BLE}{name=BLE, description={Bluetooth Low Energy ist eine Technologie, die Teil des Industriestandards Bluetooth ist und eine energiesparende, kabellose Kommunikation zwischen Geräten in einer Entfernung von bis zu ca. 10 Metern ermöglicht.}}
\newglossaryentry{Wearable Computer}{name=Wearable Computer, description={Unter dem Begriff Wearable Computer versteht man Computersysteme, die am Körper, unter der Kleidung oder als Implantat unter der Haut getragen werden können.}}
\newglossaryentry{IMU}{name=6-Achsen IMU, description={Ein 6-Achsen IMU ist ein Beschleunigungssensor mit Gyroskop.}}
\newglossaryentry{GUI}{name=GUI, description={GUI ist die Abkürzung für den englischen Begriff \glqq graphical user interface\grqq . Sie ist die Schnittstelle zwischen Mensch und Maschine und ermöglicht dem Nutzer die Eingabe/Steuerung, der Maschine.}}
\newglossaryentry{CPB}{name=Cross-Platform Bibliothek, description={Eine Cross-Platform Bibliothek ist nichts weiter als eine Bibliothek die auf Rechnersystemen mit verschiedener Architektur laufen kann.}}
\newglossaryentry{Steuerungsparameter}{name=Steuerungsparameter, description={Unter Steuerungsparametern fassen wir die Länge der Verbindungsintervalle zwischen Earables und Smartphone sowie die Abtastrate und den Wertebereich von integriertem Gyroskop und Beschleunigungssensor zusammen.}}
\newglossaryentry{Schrittfrequenz}{name=Schrittfrequenz, description={Die Schrittfrequenz gibt an wie viele Schritte pro Zeiteinheit gemacht werden.}}
\newglossaryentry{Rohdaten}{name=Rohdaten, description={Als Rohdaten werden unverarbeitete Daten bezeichnet.}}
\newglossaryentry{TTS}{name=Text-To-Speech, description={Ein Text-to-Speech-System (TTS) (oder Vorleseautomat) wandelt Fließtext in eine akustische Sprachausgabe um. Dabei erfolgt diese auf Deutsch oder auf Englisch, abhängig davon, was als Sprache eingestellt ist.}}
\newglossaryentry{Earables}{name=Earables, description={Eine Zusammenschließung des Wortes Wearable und Earphone. Dabei handelt es sich um Kopfhörer, die mit Sensoren ausgestattet sind.}}
\newglossaryentry{Vorgangsdaten}{name=Vorgangsdaten, description={Daten, die bei der Ausführung eines Vorgangs gespeichert werden (z.B. Schritte, Sit-ups,\dots).}}
\newglossaryentry{Datenbank}{name={Datenbank},
	description={Die Datenbank speichert die Trainingsdaten in Form von DBEntries. Dabei wird die Datenbank von dem Plugin SQLite benutzt.}}
\newglossaryentry{SQLite}{name={SQLite-net-pcl},
	description={SQLite-net-pcl basiert auf dem PlugIn SQLite und ist eine portable Version für die Arbeit mit einer SQLite Datenbank. Dabei ist SQLite eine Erweiterung, mit der man lokale Datenbanken ansprechen und verwalten kann. Dabei bietet die Erweiterung eine lokale Datenbank platformunabhängig gleich anzusprechen. Diese Erweiterung wird für die Speicherung der Trainingsdaten benutz}}
\newglossaryentry{CSV}{name={CSV},
	description={CSV steht für "Comma-Seperated-Value" ein Dateityp, bei dem die Attribute durch ein Komma getrennt werden. Eine Zeile bildet dabei immer einen Eintrag ab. Das Format ist human-readable und veränderbar.}}
\newglossaryentry{Rg.Plugins.Popup}{name={Rg.Plugins.Popup},
	description={Rg.Plugins.Popup ist eine Erweiterung (Plugin), welche es einem ermöglicht Pop-up Fenster zu erstellen und anzuzeigen. Dabei ist das Plugin über den Pluginmanager 'NuGet' einbindbar.}}
\newglossaryentry{DependencyInjection}{name={Microsoft.Extensions.DependencyInjection},
	description={Microsoft.Extensions.DependencyInjection ist eine Erweiterung, welches ermöglicht Komponenten als Services zu registrieren und sie somit abgekapselt und modularer zu betrachten. Mit Klassen, wie der ServiceCollection, oder dem ServiceProvicer, bietet die Erweiterung Microsoft.Extensions.DependencyInjection das Framework.}}
\newglossaryentry{NuGet}{name={NuGet},description={NuGet ist ein Plugin-manager für Visual Studio. Dieser ermöglicht es einem einfach neue Erweiterungen dem Projekt hinzuzufügen. Er kümmert sich um die Kompabitlität und bietet ein Verzeichnis der unterschiedlichen Plug-ins.}}
\newglossaryentry{Bluetooth LE Plugin}{name={Bluetooth LE Plugin for Xamarin},description={Diese Erweiterung ermöglicht es die Verbindung mit den \Gls{Earables} herzustellen. Zudem wird die Kommunikation zwischen Device und \Gls{Earables} geregelt. Von dieser Erweiterung erhalten wir die \Gls{IMU}-Daten.}}
\newglossaryentry{Xamarin.Essentials}{name={Xamarin.Essentials},description={Die Erweiterung Xamarin.Essentials ermöglicht einem, Devicespezifische Services anzusprechen. Benutzt wird hiervon die Funktion TextToSpeech. }}
\newglossaryentry{Xamarin.Plugin.Filepicker}{name={Xamarin.Plugin.Filepicker},description={Die Erweiterung Xamarin.Plugin.Filepicker ergänzt die Funktionalität eines Dateiauswähler. Mit diesem kann man in den Dateimanager des Devices und Dateien auswählen. Gebraucht wird dies beim Importieren und Exportieren der Trainingsdaten.}}

\end{document}
